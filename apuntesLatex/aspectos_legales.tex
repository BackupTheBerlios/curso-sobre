% Capítulo incluido en los Apuntes Colaborativos realizados por los
% alumnos de Software Libre en el Máster Universitario Oficial en
% Sistemas Telemáticos e Informáticos (Universidad Rey Juan Carlos),
% durante el curso 2010/11.

\chapter{Aspectos Legales}
En este capítulo se verán los elementos principales involucrados en
los aspectos legales del software libre. El más importante de ellos es
la licencia, que supone el contrato establecido entre el autor y los
usuarios y que define qué libertades tienen estos últimos sobre la
obra en cuestión. Antes de entrar en detalle con las licencias se dará
un barrido sobre la propiedad intelectual, que engloba conceptos como
\emph{Copyright} y patente, entre otros.

\section{La propiedad intelectual}
La propiedad intelectual engloba los derechos de autor, las patentes,
el secreto industrial y las marcas. Es compleja, principalmente por la
dificultad de marcar sus límites. Prueba de ello es el caso del puente
Zubi~Zuri bilbaíno, en el que Santiago Calatrava, diseñador del
puente, denunciaba las modificaciones que el ayuntamiento de la ciudad
efectuó sobre su obra. En una primera instancia, la reclamación del
pago de una indemnización de 250.000€ fue denegada. Tras la
presentación de un recurso, el arquitecto recibió 30.000€. Esta
sentencia deja la sensación de que el bien público se antepone sobre
la propiedad intelectual. Hay otros casos interesantes como la demanda
que el artista Nach interpuso sobre el Ministerio~de~Sanidad, motivado
por un anuncio que fomentaba el uso del preservativo. Tanto el tema
``Efectos~Visuales'' del cantante como la canción del anuncio
compartían estilo músical y versos en los que se utilizaba una única
vocal. La petición fue desestimada. El omnipresente Google también ha
sufrido demandas de este tipo. Su programa de recopilación de noticias
Google~News hacía que los periódicos autores de las noticias perdieran
ingresos al disminuir los interesados en publicitarse en sus páginas
oficiales. Como se puede observar, la propiedad intelectual afecta a
muchos sectores. En las siguientes secciones veremos cómo nace el
concepto, las diversas interpretaciones geográficas y las prácticas
habituales para salvaguardarla.

\subsection{Orígenes de la propiedad intelectual}
La propiedad intelectual nace junto con uno de los mayores inventos de
la historia: la imprenta. Hasta ese momento, la copia manual era la
única forma de reproducción de obras pósible. Los autores ni siquiera
se habían planteado el problema que vendría años después en
Inglaterra.

La imprenta ofrecía la posibilidad de reproducir libros con gran
rapidez, lo que desembocó en nuevos modelos de negocio. La venta de
libros y periódicos a gran escala se convertía así en algo
factible. Los dueños de las imprentas empezaron a rentabilizar sus
máquinas. Los autores, conscientes del nuevo mercado, exigían su parte
de beneficios. Para ello tenían que llegar a acuerdos con los
impresores. Pero, \emph{¿qué pasaba si otra imprenta se hacía con una
  copia y la reproducía sin llegar a un acuerdo con el autor?} Para
solventar ese problema, los ingleses realizaron ``pactos de
caballeros'' mediante los que no se publicaría ninguna obra sin el
consentimiento del escritor. Pronto llegarían los galeses para
saltárselos.

Los empresarios ingleses empezaron a manifestar su animadversión
contra sus competidores galeses. Ellos no pagaban a los autores, por
lo que su margen de beneficios era mayor. Para hacer frente al
problema, acudieron a la Reina~Ana~de~Gran~Bretaña, máxima autoridad
de la época, y expresaron sus quejas. La Reina encontró una solución
para los impresores ingleses y sobre todo para sí misma: impuso un
permiso de copia que sólo podía ser otorgado por la Corona. Con esta
decisión adquiere un control total sobre los libros que se imprimían,
que podrían contener ideas subversivas contra el reino. Nace así la
censura en las imprentas.

Más tarde, surge en Francia una nueva rama de la propiedad
intelectual. Consiste en la protección del autor por encima del resto
de factores. Supone que el artista es tocado por Dios, lo que le
permite crear su obra. Esta rama ha tenido una gran influencia en
nuestra legislación actual\footnote{En palabras de Grex: ``¡nuestros
  artistas están afrancesados!''}, en la que se separa
\emph{propiedad~intelectual} de \emph{propiedad~industrial}. Una vez
introducido el origen del Copyright, pasamos a describir elementos
vigentes hoy día como las patentes, el secreto comercial y las
marcas.

\subsection{Las patentes}

\subsection{El secreto comercial}

\subsection{Las marcas}

\section{Licencias}

