%
% Capítulo incluido en los Apuntes Colaborativos realizados por los
% alumnos de Software Libre en el Máster Universitario Oficial en
% Sistemas Telemáticos e Informáticos (Universidad Rey Juan Carlos),
% durante el curso 2010/11.
%

\chapter{Aspectos Legales}
\label{CHAP2:LegalAspects}
En este capítulo se verán los elementos principales involucrados en
los aspectos legales del software libre. El más importante de ellos es
la licencia, que supone el contrato establecido entre el autor y los
usuarios y que define qué libertades tienen estos últimos sobre la
obra en cuestión. Antes de entrar en detalle con las licencias se dará
un barrido sobre la propiedad intelectual, que engloba conceptos como
\emph{Copyright} y patente, entre otros.

\section{La propiedad intelectual}
La propiedad intelectual engloba los derechos de autor, las patentes,
el secreto industrial y las marcas. Es compleja, principalmente por la
dificultad de marcar sus límites. Prueba de ello es el caso del puente
Zubi~Zuri bilbaíno, en el que Santiago Calatrava, diseñador del
puente, denunciaba las modificaciones que el ayuntamiento de la ciudad
efectuó sobre su obra. En una primera instancia, la reclamación del
pago de una indemnización de 250.000 euros fue denegada. Tras la
presentación de un recurso, el arquitecto recibió 30.000 euros. Esta
sentencia deja la sensación de que el bien público se antepone sobre
la propiedad intelectual. Hay otros casos interesantes como la demanda
que el artista Nach interpuso sobre el Ministerio~de~Sanidad, motivado
por un anuncio que fomentaba el uso del preservativo. Tanto el tema
``Efectos~Visuales'' del cantante como la canción del anuncio
compartían estilo músical y versos en los que se utilizaba una única
vocal. La petición fue desestimada. El omnipresente Google también ha
sufrido demandas de este tipo. Su programa de recopilación de noticias
Google~News hacía que los periódicos autores de las noticias perdieran
ingresos al disminuir los interesados en publicitarse en sus páginas
oficiales. Como se puede observar, la propiedad intelectual afecta a
muchos sectores. En las siguientes secciones veremos cómo nace el
concepto, las diversas interpretaciones geográficas y las prácticas
habituales para salvaguardarla.

\subsection{Orígenes de la propiedad intelectual}
La propiedad intelectual nace junto con uno de los mayores inventos de
la historia: la imprenta. Hasta ese momento, la copia manual era la
única forma de reproducción de obras pósible. Los autores ni siquiera
se habían planteado el problema que vendría años después en
Inglaterra.

La imprenta ofrecía la posibilidad de reproducir libros con gran
rapidez, lo que desembocó en nuevos modelos de negocio. La venta de
libros y periódicos a gran escala se convertía así en algo
factible. Los dueños de las imprentas empezaron a rentabilizar sus
máquinas. Los autores, conscientes del nuevo mercado, exigían su parte
de beneficios. Para ello tenían que llegar a acuerdos con los
impresores. Pero, \emph{¿qué pasaba si otra imprenta se hacía con una
  copia y la reproducía sin llegar a un acuerdo con el autor?} Para
solventar ese problema, los ingleses realizaron ``pactos de
caballeros'' mediante los que no se publicaría ninguna obra sin el
consentimiento del escritor. Pronto llegarían los galeses para
saltárselos.

Los empresarios ingleses empezaron a manifestar su animadversión
contra sus competidores galeses. Ellos no pagaban a los autores, por
lo que su margen de beneficios era mayor. Para hacer frente al
problema, acudieron a la Reina~Ana~de~Gran~Bretaña, máxima autoridad
de la época, y expresaron sus quejas. La Reina encontró una solución
para los impresores ingleses y sobre todo para sí misma: impuso un
permiso de copia que sólo podía ser otorgado por la Corona. Con esta
decisión adquiere un control total sobre los libros que se imprimían,
que podrían contener ideas subversivas contra el reino. Nace así la
censura en las imprentas.

Más tarde, surge en Francia una nueva rama de la propiedad
intelectual. Consiste en la protección del autor por encima del resto
de factores. Supone que el artista es tocado por Dios, lo que le
permite crear su obra. Esta rama ha tenido una gran influencia en
nuestra legislación actual\footnote{En palabras de Grex: ``¡nuestros
  artistas están afrancesados!''}, en la que se separa
\emph{propiedad~intelectual} de \emph{propiedad~industrial}. Una vez
introducido el origen de las prácticas, pasamos a ver uno de los
elementos que más perjudica al Software Libre, las patentes.

\subsection{Las patentes}
Una patente otorga el monopolio del invento a su creador durante un
periodo que va desde 17 a 25 años, según la región. El inventor ha
dedicado un tiempo al trabajo y recibe esta recompensa. A cambio, el
invento debe ser públicado. Las terceras partes que quieran trabajar
sobre él deben llegar a un acuerdo con el titular de la patente.

No se puede patentar una idea. La institución registradora de patentes
se encarga de acotar los límites del invento. \emph{La patente de una
  silla, ¿debería permitirme explotar también sillones y sofas?} Sin
embargo, hay organizaciones como la \emph{Organización Mundial de la
  Propiedad Intelectual} (OMPI) que están ejerciendo presión para que
se relajen las restricciones de registro de una patente. No es un
procedimiento al alcance de cualquiera por lo que unas pocas empresas
suelen ser dueñas de la inmensa mayoría de patentes. Por otro lado, se
ha detectado una práctica denominada \emph{patent trolling} que
engloba a las entidades que se encargan de patentar el mayor número
posible de inventos, no para explotarlos y lucrarse con ellos, sino
para obtener beneficios mediante los acuerdos con interesados en el
invento.

En el software libre las patentes suponen un problema muy grande. El
acceso al código fuente facilita la detección de infracciones sobre
patentes. Los autores de software privativo tienen más ventajas en
este sentido. En Europa, a diferencia de Estados Unidos, todavía no es
posible patentar software, aunque podría llegar ese momento.

\subsection{El secreto comercial}
Las empresas tienen el derecho a no hacer público su trabajo. Un
ejemplo muy claro de este tipo de práctica es la ocultación de la
fórmula de la \emph{Coca-Cola}. A la compañía no le interesa una
patente, ya que saben que pueden rentabilizar su producto mucho más
allá de la vigencia de ésta. Otro ejemplo es la ocultación de la
ingeniería utilizada entre las diferentes escuderías de la
\emph{Fórmula 1}, donde el prestigio por ganar el campeonato tiene un
peso muy grande. En ambos casos se pretende esconder los secretos de
la empresa para dificultar el trabajo a los competidores.

El secreto comercial resulta más limpio que una patente, en términos
de mercado. La razón es que permite la coexistencia con terceras
partes que se dediquen a imitar el producto original. No hay más que
ver las infinitas variedades de refrescos de cola que se pueden
adquirir, habiéndose convertido \emph{Pepsi} en el adversario más
fuerte. En la imitación tiene una gran importancia la ingeniería
inversa, que permite deducir el proceso de manufacturación a partir
del producto final. Los secretos mejor guardados se encuentran en las
grandes compañías, que como viene siendo habitual, ejercen mucha
presión sobre el Estado cuando se ven amenazados. Por eso, existen
regiones en las que la ingeniería inversa está prohibida.

En el software, el proceso de creación de un producto es encarnado por
el código fuente. Lo que permite el secreto comercial es la
distribución del binario. Por lo tanto, está práctica carece de
sentido en el software~libre, ya que aplicarla iría en contra de las
libertades básicas al no disponerse del fuente.

\subsection{Las marcas}
Una marca es el nombre ---y opcionalmente el logotipo--- que utiliza
un negocio para promocionarse. Registrar una marca requiere cierto
desembolso económico, por lo que generalmente no suele
realizarse. Esto explica en cierta manera por qué es posible que
existan cientos de bares ``Pepe'', pero dificilmente se encontrará una
tienda deportiva que utilice el nombre ``Nike''.

En el software~libre, son sólo los grandes productos y compañías los
que las utilizan. Entre ellos encontramos a Debian, GNOME o GNU. En la
historia del software~libre se han encontrado problemas por el
no~registro de una marca. El caso más representativo es el de Linux,
en el que varias personas registraron esa marca y reclamaron royalties
a las distintas distribuciones existentes. También es un problema de
marcas el que tuvieron Firefox y Debian. Esta distribución se negó a
incluir el navegador por la falta de libertad sobre los logos
oficiales y el nombre de Firefox. La solución fue incluir el navegador
con el nombre y el logotipo renombrados, dando como resultado el
proyecto IceWeasel.

\subsection{Derechos de autor}
Los derechos de autor surgen para recompensar a los autores de libros
o de arte. Representan lo que también es conocido como
\emph{Copyright} y tratan de proteger la expresión de un contenido, no
el contenido en sí mismo. Por dar un ejemplo actual, todos los días
vemos en los periódicos artículos que tratan sobre los mismos sucesos,
pero contados de forma diferente. Si se protegiese el contenido, la
noticia sólo podría ser publicada en un periódico.

En España, la encargada de estos derechos~de~autor es la denominada
Ley~de~Propiedad~Intelectual. Esta ley hace una subdivisión en
Derechos~Morales y Derechos~Patrimoniales. Los primeros hacen posible
la autoría de una obra y que se respete su integridad. Son vitalicios
o indefinidos. Los segundos son los que permiten al autor la
explotación económica de su trabajo. Tienen un periodo de vigencia de
70 años\footnote{Otro ejemplo de presión sobre la legislación de la
  propiedad intelectual es la que ejerce
  Mickey~Mouse. Misteriosamente, la ley se prolonga cuando este
  simpático ratón está a punto de pasar al dominio
  público.}. Actualmente, la forma de obtener los derechos~de~autor es
automática. En cuanto se crea una obra, entra en vigor. El autor de un
garabato en una pizarra dispone de todos los derechos sobre él. Hace
años no se aplicaban por defecto. Es curioso el caso de la mítica
película de título \emph{La noche de los muertos vivientes} en la que
olvidaron incluir la (C) de Copyright, por lo que directamente pasó a
formar parte del dominio público.

\subsection{¿Qué pasa con el Software?}
\label{sub:que_pasa}
Hasta el momento hemos visto diversos mecanismos para tratar la
propiedad~intelectual. \emph{¿En cual de ellos encaja el Software?} En
un principio hubo mucha polémica por decidir si se debían aplicar
patentes o derechos de autor, finalmente se impuso el segundo de
ellos. No resultaría mantenible solicitar una patente por cada commit
realizado. Además existían algunas similitudes con las novelas y dicha
legislación había funcionado bien hasta el momento. No tardarían en
aparecer incongruencias y situaciones especiales. Y es que el software
tiene unas características que le hacen muy diferente del resto de
obras existentes hasta la fecha.

\section{Licencias}

% Sólo se listan las licencias que vimos en clase. Se podría completar
% con muchas más.

Si navegando por \emph{la Red} descubrimos un software sin licencia,
\emph{¿qué podríamos hacer con él?} Acabamos de ver en
\ref{sub:que_pasa} que lo que aplicamos al software son los
derechos~de~autor, es decir, la obra automáticamente pasa por defecto
a restringir todos los derechos de los usuarios. Por tanto, si no
encontramos la licencia en dicho programa, debemos asumir que no
podemos hacer absolutamente nada con él. Bien es cierto que sería
difícil que un juez declarara culpable a una persona por usar ese
software. Sin embargo, la modificación y redistribución ya sí que
sería un terreno muy pantanoso.

Entonces, \emph{¿qué es una licencia?} Es un contrato mediante el cual
el autor se comunica con los usuarios finales. Aunque esto a veces es
difícil de entender, la licencia no restringe, sino que se encarga de
otorgar libertades. Sin ella, comprar un programa no nos daría la
capacidad de usarlo tan siquiera, se debe recordar lo que se comentó
en el párrafo anterior. La licencia es un contrato muy especial, un
contrato que ni firmamos ni aceptamos explícitamente. En ocasiones el
instalador pide la confirmación mediante la típica checkbox
\emph{I~Agree}, pero hay casos en los que el entorno no permite esta
posibilidad, como por ejemplo el software embebido en una lavadora.

Las licencias son textos largos y complicados. \emph{¿Qué pasa si no
  entendemos alguna en concreto?} Siempre que se produce un acuerdo,
ambas partes deben ser conscientes de las implicaciones que
conlleva. Por esa razón los bancos encuentran muchos problemas con sus
clientes, hasta tal punto que algunos acuerdos requieren que el
interesado tenga un cierto nivel educacional. En el caso del software
la solución es más sencilla. Si no entendemos la licencia, se debe
suponer la establecida por defecto, con lo que no se tendría derecho a
nada. Otro aspecto habitual es el de encontrar una licencia en otra
lengua. Por ejemplo, la FSF sólo considera como válida la GPL en
inglés. Las traducciones pueden dar lugar a segundas interpretaciones,
a veces incluso provocadas. Ante este caso, siempre hay que quedarse
con la interpretación más restrictiva o consultar directamente al
autor si fuese posible.

Licenciar software con una licencia libre básica es sencillo. Tan sólo
debemos incluir las siguiente líneas al principio de los ficheros
fuente del programa:
\begin{verbatim}
Copyright (c) 2010 Foobar Developers. All rights reserved.

Redistribution and use in source and binary forms, with or without
modification, are permitted provided that the redistributions of
source code must retain the above copyright notice.
\end{verbatim}
Con la primera línea se está ``marcando el terreno''. En ella se puede
apreciar el año de creación, el nombre del programa y el nombre de los
autores. La segunda parte del texto es la que otorga al usuario final
la libertad de uso, redistribución, modificación y redistribución de
la modificación (principios básicos del software libre). Resulta
curioso observar como primero se reservan todos los derechos para
posteriormente concederlos. La única restricción que se impone al
usuario ---en este ejemplo particular--- es la de mantener la primera
línea tal cual, con el fin de preservar la autoría. En los siguientes
apartados se describirán las distintas familias de licencias,
nombrando para cada una sus ejemplares más representativos.

\subsection{Licencias Académicas}
Son también conocidas como licencias Minimalistas o de tipo BSD. Son
muy adecuadas para investigaciones académicas, generalmente
financiadas por el Estado. Pretenden tener un alcance global, por lo
que no desean cerrar ninguna puerta. Esto conlleva otorgar todas las
libertades al usuario final. La única condición a cambio es la de
conservar la autoría de los creadores. Un ejemplo de software en el
que se aplica esta licencia es el protocolo TCP/IP. Es un protocolo
utilizado por cualquier Sistema~Operativo moderno, tanto libre como
privado. Las libertades que ofrece esta familia rozan las que podemos
encontrar en un software de dominio~público.
\begin{description}
\item[Licencia BSD] Es la más representativa de esta familia. No
  ofrecen ninguna garantía al usuario final. Éste es una aspecto que
  siempre ha generado cierta controversia, aunque es bastante
  comprensible\footnote{Los desarrolladores de software~propietario
    tan sólo garantizan que el soporte es correcto y que el programa
    se ejecuta.}. La permisividad de esta licencia provoca que el
  software se pueda redistribuir tanto en versión de código fuente
  como en binario. Para el primer caso, se debe mantener la línea del
  Copyright y la lista de condiciones en los ficheros fuente. Con el
  formato binario se debe reproducir de alguna manera el mismo texto
  que para el caso anterior. Por ejemplo, Windows muestra la nota BSD
  durante la carga del sistema. Se puede leer la plantilla de esta
  licencia en
  \url{http://www.opensource.org/licenses/bsd-license.php}.
\end{description}

\subsection{Licencias Permisivas}

Las licencias permisivas son aquellas que no imponen ninguna
restricción en la segunda redistribución. Las licencias académicas que
veíamos en el punto anterior también corresponden a esta familia. No
obstante se suele hacer esta distinción para diferenciar las licencias
académicas, viejas y cortas, con las licencias permisivas actuales,
más modernas y más largas, pero herederas naturales de la BSD.

\begin{description}
  \item[Licencia Apache] Las licencias académicas eran muy buenas para
    aplicarlas en estándares como TCP. No obstante los desarrolladores
    del proyecto Apache detectaron que esa licencia no se acoplaba
    demasiado bien a su trabajo. El problema residía en que terceras
    partes podrían ensuciar su nombre, al realizar ``malas'' obras
    derivadas. Apache quería una licencia que otorgase libertad a los
    demás sin nombrar su proyecto como autor. \emph{Haz con Apache lo
      que quieras, pero no digas que es de Apache}. Así la Apache
    Software Foundation (ASF) crea la licencia Apache, aprobada como
    libre por la FSF y por la OSI. Es muy similar a la BSD sólo que
    añade alguna restricción, como la de autoría. Hay numerosos
    proyectos que utilizan esta licencia, donde podemos encontrar el
    25\% del código de Google.
\end{description}

Hay muchas otras licencias permisivas con principios muy similares a
la licencia Apache. Destacan la X11 y la Zope.

\subsection{Licencias Copyleft}

Esta familia de licencias, también conocidas como licencias robustas,
son las que imponen condiciones sobre la segunda redistribución. Esa
imposición suele requerir que se mantenga la misma licencia. Hay un
gran debate por si esta técnica fomenta la libertad, por estar
obligando a que las obras derivadas también sean libres, o si la
coarta, al imponer restricciones a terceras partes.

\subsubsection{Copyleft débil}

Son aquellas licencias que permiten la coexistencia con obras
privativas\footnote{Hay que tener en cuenta que cuando hablamos de
  coexistencia lo hacemos en términos de redistribución. Cualquiera
  podría juntar en una máquina software privado con software libre
  copyleft. Por poner un ejemplo, puedes instalar Emacs en
  Windows.}. No obstante, el software licenciado con copyleft débil,
debe modificarse bajo la misma licencia.

\begin{description}
\item[LGPL] Permite la coexistencia con ficheros que tengan otra
  licencia. En sus inicios fue pensada para bibliotecas, con el fin de
  fomentar algunas librerías GNU sin oponer tantas restricciones como
  la GPL, que veremos más adelante. A día de hoy la FSF ---con fuerte
  predilección por las licencias copyleft fuerte--- recomienda no
  utilizarla, por la ventaja que se otorga al sector de desarrollo
  privativo. Les permite lucrarse con bibliotecas libres para trabajos
  cuyo código no verá la luz.
\item[Mozilla Public License] Muy similar a la LGPL aunque
  incompatible con la GPL. Esto ha derivado en que Firefox, que usaba
  esta licencia originariamente, haya tenido que liberarse
  simultáneamente bajo diversas licencias, entre ellas la GPL.
\end{description}

\subsubsection{Copyleft fuerte}

No se permite la coexistencia con ficheros privativos. Al ser copyleft
obliga a que las obras derivadas mantengan la misma licencia. Con esto
se promueve el desarrollo de Software Libre.

\begin{description}
\item[GPL] La GNU General Public License es sin duda la licencia libre
  más popular. El concepto copyleft nace con ella. Aprovecha el
  copyright para otorgar las libertades básicas y sin ella, lo que
  conocemos hoy día como Software Libre sería una utopía. Supone el
  mecanismo que lo hace realidad. Tiene 3 versiones. La última de
  ellas contempla nuevas claúsulas adaptadas a la tecnología
  actual. Por ejemplo, evita el problema que existía con TiBO, un
  dispositivo sobre el que corría software libre. El hardware
  rechazaba cualquier modificación en el código, ya que tenía escrita
  la firma original del mismo a fuego. Elimina alguna restricción
  sobre la GPL2, que permite la compatibilidad con licencias como
  Apache2. Recoge también alguna claúsula referente a las patentes. Si
  alguien, autor de una patente, decide utilizarla en un programa
  libre, debe otorgar un permiso de uso a cualquier persona interesada
  en modificar la nueva obra.
\item[GPL Affero] Es una variante de la GPL para el problema surgido
  con \emph{la nube}. Si alguien se instala un programa con licencia
  GPL es libre de hacer lo que quiera con él en su máquina local, sin
  necesidad de compartir el código fuente, ya que no lo está
  distribuyendo. Pero, \emph{¿qué pasa si ese software es un Gestor de
    Contenidos (CMS) o un Servicio Web al que se le han aplicado unas
    modificaciones? ¿Ofrecer dichos servicios supone una
    redistribución?}  En principio la respuesta es que no. Por ello,
  surge la licencia Affero, para obligar a compartir el código fuente
  con las modificaciones en casos especiales como este. Servicios como
  Meneamé están bajo esta licencia.
\end{description}

\subsection{Notas finales}

La existencia de múltiples licencias no es algo positivo. Comprender
las implicaciones de cada una resulta complicado. Por ello, es muy
recomendable utilizar las más representativas con el objetivo de no
confundir a los usuarios finales. Las licencias que se han recogido en
las secciones anteriores están muy extendidas y por tanto son
adecuadas.

Elegir la licencia que más convenga a un proyecto es muy
complicado. Si MySQL hubiera utilizado Apache en vez de GPL, no
hubiese alcanzado los grandes ingresos que obtuvo al ser
vendida. Equivocarse de licencia también puede llevar a la ruina.

Otro aspecto que se debe de tener muy en cuenta es la compatibilidad
entre licencias. Haciendo una analogía con el mundo real, podemos
encontrar dos chicos que desean montar una fiesta en casa. Uno de
ellos es más permisivo y el otro pondrá muchas pegas a lo que se puede
o no hacer en la casa. Evidentemente, el liberal tendrá que ceder ante
las exigencias de su compañero. En el mundo de las licencias de
software libre, el compañero restrictivo será sin duda alguna el que
represente a la GPL. Utilizar esta licencia implica que las obras
derivadas sean siempre GPL. Esto genera cierta controversia entre los
desarrolladores de software libre. \emph{¿Es más libre la total
  libertad u obligar a que todo sea siempre libre?}

\section{Otras obras libres}

El Software Libre se ha convertido en un referente para otros sectores
de la sociedad. Su éxito ha provocado la migración del espíritu, tanto
ético como pragmático, a otras obras intelectuales. Si realizásemos
una novela, licenciarla con GPL no tendría demasiado
sentido. Confundiría a otros artistas, que no entienden de
informática. Además, los principios que se pueden aplicar al software,
un tipo de obra único, no suelen ser factibles en otros
entornos. Richard M. Stallman clasificó las obras en
\emph{funcionales} y \emph{no funcionales}. El software se encuentra
en el primer conjunto. Era necesario crear la infraestructura para
poder hacer libres las obras del segundo grupo.

Richard M. Stallman solía poner licencias muy restrictivas sobre sus
documentos, con el fin de que nadie tergiversara sus palabras. Por
ello se planteó si las obras no funcionales requerían restricciones
adicionales sobre las funcionales. Por otro lado, surgió el debate
sobre la inclusión de claúsulas comerciales. Finalmente, un
desarrollador de Debian, basándose en las cuatro libertades de la FSF,
definió qué condiciones debía tener una obra para pertenecer a la
Cultura Libre. El resultado, que podemos encontrar en
\url{http://freedomdefined.org/}, deja fuera a los trabajos con
restricciones comerciales. En las siguientes secciones veremos las
licencias más habituales.

\subsection{GNU Free Documentation License}

Richard M. Stallman detectó que la documentación del software no
encajaba bien con la licencia aplicada al código fuente. También se
dio cuenta de que al igual que en el software, en la documentación
existían ``fuentes'' y ``binarios''. Correspondían con las copias
transparentes (fichero Latex, formato de páginas de manual...) y
copias opacas (pdf, ps...), respectivamente. Identificó que algunas
partes del documento, tales como agradecimientos y dedicatorias, no
debían ser modificadas, aunque sí que debería permitirse añadir nuevas
líneas. Finalmente observo que algunas partes del documento, las no
técnicas, debían de ser invariantes. Juntando todas estas ideas se
crea la GNU FDL.

La licencia tenía retractores, que veían en las partes ``invariantes''
una limitación de la libertad. Debian, muy comprometido con sus
usuarios en este aspecto, tuvo un gran debate interno para decidir su
uso. Finalmente acordaron que si no se aplicaban dichas partes
invariantes era una licencia libre. Un inconveniente asociado a la FDL
es que se debe incluir el texto completo ---unas treinta páginas--- al
final del documento\footnote{Aplicar esta práctica en un documento muy
  breve deriva en una obra extraña y poco equilibrada \emph{¡Dos caras
    de contenido y 30 de licencia!}}. Esta licencia ha sido
inspiración para una nueva generación de licencias aplicadas a otras
obras. Veremos las Creative Commons en \ref{sub:creative}.

\subsection{Creative Commons}
\label{sub:creative}
Creative Commons nace con la idea de crear un conjunto de licencias
para cualquier tipo de contenido, que sean fácilmente
reconocibles. Sus creadores son abogados, hecho que se manifiesta en
el enfoque a la hora de publicar las licencias. Usan una especie de
``ingeniería de licencias'': la escriben, la publican y estudian cómo
se comporta en la sociedad. Elaboran licencias\footnote{Una curiosidad
  de Creative Commons es que la misma versión de una licencia puede
  ser diferente en un país que en otro.} cercanas al dominio público
pero con algunas restricciones basadas en la demanda de los
autores. Detectaron el derecho a atribución, restricción de obras
derivadas comerciales, restricción de obras derivadas y el share-alike
(compartir igual) como elementos principales. Si el eslógan de los
derechos de autor era \emph{All rights reserved} y el del copyleft
\emph{All rights reversed}, Creative Commons utiliza \emph{Some rights
  reserved}.

Aunque con los derechos más demandados se podrían usar muchas
combinaciones para crear licencias, son sólo 6 las que se suelen
utilizar:
\begin{enumerate}
\item Attribution (by)
\item Attribution + Non commercial (by-nc)
\item Attribution + No Derivate Works (by-nd)
\item Attribution + share alike (by-sa)
\item Attribution + Non commercial + No Derivate Works (by-nc-nd)
\item Attribution + Non commercial + share alike (by-nc-sa)
\end{enumerate}
Se tiende a pensar en las licencias Creative Commons como libres, pero
lo cierto es que únicamente la primera y la cuarta lo son. Restringir
las obras derivadas o el comercio va en contra de la Cultura
Libre. Por ello, sólo las licencias que comentábamos tienen el sello
de \emph{Free Cultural Works}. Por otro lado, hay que tener cuidado
con afirmar rotundamente que share-alike y copyleft se refieren
exactamente al mismo concepto. Si bien es cierto que la idea es la
misma, segundas redistribuciones con las mismas condiciones
originales, el copyleft lleva implícita la libertad. Por su parte, el
share-alike puede mezclarse con claúsulas no comerciales, lo que
supondría redistribuciones no libres.

Existen otras licencias Creative Commons algo especiales. En clase
vimos las siguientes:
\begin{description}
\item[Creative Commons Zero] El concepto de Dominio Público puede ser
  diferente en cada país. Esta licencia pretende encarnarlo sin la
  inseguridad jurídica regional.
\item[Founder's Copyright] Al igual que existen las canciones
  protesta, esta es sin duda una licencia protesta. Hace que a los 14
  años desde la creación de la obra, ésta pase a dominio
  público. Nadie la utiliza.
\item[Sampling Plus] Para temas musicales, no entraremos en detalle.
\item[Developing Nations] Esta licencia, ya retirada, resultaba
  realmente curiosa. Pretendía comportarse como una licencia libre en
  el tercer mundo y como privativa en sociedades más avanzadas. En
  Internet no hay geografía, lo que supuso el principal causante de su
  desaparición.
\end{description}

Para facilitar su extensión, las CC suelen encontrarse en tres
formatos diferentes. El primero es el texto completo ``para
abogados'', que es el que tiene validez jurídica. El segundo es un
formato más sencillo ``para humanos'', un resumen comprensible por
cualquier persona. Finalmente hay un texto ``para máquinas'' que a
modo de metainformación permite la interacción con buscadores como
Google\footnote{La sección \emph{Derechos de uso} de la búsqueda
  avanzada de Google permite filtrar los resultados por los derechos
  que otorga la licencia de la obra.}.

Para finalizar, es importante incidir en que las licencias
no-comerciales no son libres y se suelen utilizar con cierto
desconocimiento. Los autores tienden a pensar \emph{no quiero que otro
  se forre con mi trabajo} pero lo cierto es que usar esta restricción
suele ir en contra de los beneficios del propio autor. La línea que
distingue lo que es comercial de lo que no lo es es muy difícil de
trazar. Por ejemplo, una canción licenciada con CC-nc no se podría
reproducir ni en la radio, ni en bares, ni siquiera en una peluquería,
lo que supone un gran impedimento para que pueda convertirse en un
éxito. 

A pesar de algunos inconvenientes que hemos ido comentando, Creative
Commons ha supuesto un gran avance para la Cultura Libre. Su mayor
aporte ha sido el de crear un repertorio de licencias con textos
comprensibles e iconos que permiten la rápida identificación.
