\chapter{An\'alisis de proyectos}
\section{Introducci\'on}
\paragraph{Los proyectos de software libre son p\'ublicos por naturaleza y podemos tener una serie de datos que podemos encontrar y comparar con otros proyectos.}
\paragraph{Como todo an\'alisis emp\'irico tenemos que seguir una metodolog\'ia:}
\subparagraph{1. El proyecto que queremos estudiar.}
\subparagraph{2. Identificar la fuente de datos. Seguir el sistema de gesti\'on de incidencias, recuperar los datos, y una vez que hemos realizado la gesti\'on de incidencias y la recuperaci\'on de datos tenemos que realizar la miner\'ia de datos, es decir, realizar una t\'ecnica descriptiva de un proyecto.}
\subparagraph{3. Realizar el an\'alisis. Extraecci\'on la informaci\'on que no es evidente, aport\'andonos m\'as informaci\'on. Por ejemplo, en que meses del a\~no se realiza m\'as trabajos en el proyecto.}
\subparagraph{4. Realizar un informe explicativo. Una vez realizado nuestro an\'alisis tenemos que componer un informe que explica toda la informaci\'on del proyecto.}
\section{Herramientas para el an\'alisis}
\paragraph{Para realizar un an\'alisis de proyecto necesitamos un conjunto de herramientas autom\'aticas para evitar errores, para reducir tiempos, y para la replicabilidad. La replicabilidad nos permite construir un proyecto a partir de otro y no empezar de cero ahorr\'andonos un valioso tiempo. 
Es deseable que las herramientas que vamos a utilizar sean software libre porque es m\'as f\'acil de integrarlas, extenderlas y aplicarla a nuevas funcionalidades. 
Para realizar cualquier tipo de an\'alisis estad\'istico tenemos una librer\'ia de R. R es un software libre que sirve para el análisis estad\'istico y gr\'afico en un entorno de programaci\'on. Adem\'as nos permite cargar diferentes bibliotecas o paquetes con finalidades espec\'ificas de c\'alculo o gr\'afico. 
Por otro lado necesitamos soporte de administraci\'on de herramientas estad\'isticas, esto se traduce a tener un ordenador potente o un servidor potente (gran capacidad de disco duro, ram, cpu potente, etc).}
\paragraph{Qu\'e podemos usar o aplicar para realizar un an\'alisis:}
\subparagraph{1. Vamos a utilizar una base de datos, MySQL/SQLite.}
\subparagraph{2. Una herramienta que extrae la informaci\'on de c\'odigo fuente de los registros y la almacena en una base de datos, CVSAnalY.}
\subparagraph{3. Y un analizador estad\'istico, GNU R.}
\paragraph{Para la estaci\'on de datos vamos a extraer informaci\'on de repositorios p\'ublicos como son CVS, SVN y GIT. Nos proporciona un mecanismo autom\'atico que apuntando a una url nos trae la informaci\'on, la parsea y la guarda en una base de datos o en un fichero para poder trabajar.}
\paragraph{Podemos extraer datos para las acciones de desarrollo, lo vamos a tener en una tabla de la base de datos "scmlog"
Tenemos datos registrados de archivos que est\'a en la tabla "file".
Otra tabla para los datos de las personas involucradas en el proyecto "people".}
\section{Algunos an\'alisis}

