\chapter{El desarrollador y sus motivaciones}
\label{CHAP3:Developer}
\section{Introducción}

Existen muchos paradigmas respecto a las personas que están detrás del desarrollo de un proyecto de \textit{Software Libre} y sobre las razones por las cuales estas personas han decidido dedicar parte de su tiempo a esta labor, más aún cuando esta actividad en la mayoría de los casos no tiene retribuciones económicas directas y no estimula el reconocimiento público a los desarrolladores. Estos paradigmas han generado creencias falsas con poco fundamento respecto a quienes están detrás del desarrollo del \textit{Software Libre}, imaginándolos como personas fuera del común, jóvenes asociales, solitarios, “nerds”, “geeks” o inclusive como dioses o seres inmaculados con inteligencia sobrenatural a quienes no se puede mirar, hablar, ni tocar.

Para aclarar un poco este panorama nos basaremos en varios estudios científicos realizados en los últimos años y especialmente en el resultado de una investigación de la \textit{Universidad Técnica de Berlín}\footnote{ http://widi.berlios.de/paper/study.html } sobre la distribución geográfica y los antecedentes personales de los desarrolladores de \textit{Software Libre}.

El análisis de estos estudios nos permitirá conocer un poco más a fondo y con cifras reales al tipo de personas que desarrollan \textit{Software Libre} y los motivos que los llevan a dedicar tiempo y esfuerzo a esta actividad. De hecho, estos estudios fueron de especial interés para los propios miembros de las diferentes comunidades de \textit{Software Libre}, ya que si bien existe una estrecha comunicación a través de listas de correo electrónico, \textit{IRC} y demás herramientas (en definitiva, a través de la red), lo habitual es que esas personas no se conozcan físicamente. Veremos cómo los perfiles de estas personas son muy variados, que son personas normales entre hombres y mujeres de diferentes edades, profesiones, nacionalidades, ocupaciones y expectativas.

\section{Edad de los desarrolladores}

No existen restricciones de edad para contribuir en un proyecto de \textit{Software Libre}, se pueden encontrar desde adolescentes de 13 ó 14 años, hasta personas mayores de 70 años. Sin embargo el rango de edad con mayor afluencia se encuentra entre los 21 y 24 años, siendo los 23 la edad que más predomina entre los desarrolladores. 

El promedio total se sitúa en los 27 años, lo cual indica que aunque el grupo predominante está entre los jóvenes, también existe un número considerable de desarrolladores entre las personas adultas. 

En conclusión puede observase que aproximadamente el 50\% cuenta con jóvenes de 21 a 28 años, el 20\% con jóvenes y adolecentes menores de 20 años y el 30\% restante son adultos mayores de 28 años.

Otro aspecto importante a tener en cuenta es la edad de incorporación de los desarrolladores en proyectos de \textit{Software Libre}. Como era de esperarse después de ver los resultados anteriores, las edades con mayor índice de incorporación están entre los 19 y 23 años, con un fuerte  ascenso a partir de  los 18 y un fuerte descenso a partir de los 26.

Todo esto indica por lo tanto que se trata más bien de un movimiento universitario tal y como se analizará en el siguiente apartado, y no tanto de un movimiento adolescente como habitualmente se ha hecho creer.

\section{Formación académica y nivel profesional de los desarrolladores}

Es importante destacar el importante papel que cuentan las universidades en el ámbito del \textit{Software Libre}, históricamente éstas han sido los principales puntos de partida para grandes proyectos y grandes desarrolladores. De hecho, de una manera empírica se podría aproximar que de entre el total de las personas que tienen algún tipo de relación con el \textit{Software Libre}, un tercio de esas personas pertenecen a un ámbito universitario (alumnos y profesores), cosa que indica que se trata de un mundo muy académico que cuenta con una gran influencia dentro del \textit{Software Libre}, promoviéndolo, enseñándolo y sirviéndose de él como una buena manera para el aprendizaje.

Revisando el análisis realizado anteriormente sobre la edad de los desarrolladores y teniendo en cuenta que más de un 70\% de ellos cuenta con una formación universitaria, podemos observar como ese rol histórico se sigue manteniendo, pues la edad de mayor incorporación y de mayor número de desarrolladores concuerda con el promedio de edades de los estudiantes universitarios.  Para ser más exactos, podemos ver que los estudios universitarios de grado corresponden al 33\% y los estudios universitarios de máster corresponden al 28\%, lo cual constata aún más lo afirmado, pues el rango promedio de edades que corresponden a estos dos niveles académicos está entre los 18 y 26 años, justamente las edades donde existe mayor ascenso y descenso de incorporaciones. Por otro lado tenemos los estudios universitarios PhD con un 9\% (téngase en cuenta que si la moda para la edad de los desarrolladores se encuentra en los 21 años, resulta complicado encontrar personas con un doctorado a tan temprana edad), estudios escolares 19\% y otros estudios el 10\%.

En general, podemos ver que la gran mayoría de desarrolladores de \textit{Software Libre} son personas capacitadas que están cursando estudios superiores o ya cuentan con un titulo de este nivel, aunque también existe un grupo más reducido de personas empíricas que aunque no tengan estudios superiores realizan grandes aportes al \textit{Software Libre}, a la vez que en ningún momento se sienten discriminados, dado que se encuentran en un entorno que cuenta con otros mecanismos para controlar la calidad.

Entre las áreas profesionales en las que se encuentran los desarrolladores de \textit{Software Libre} podemos ver que las tecnologías de la información (IT) predominan considerablemente con un 79.69\%, cifra que a pesar de ser bastante alta nos deja muy sorprendidos, pues era de imaginar que estaría más cercana al 100\%.  Sin embargo,  encontramos un considerable 20,31\% de personas pertenecientes a otras áreas profesionales que dedican tiempo al desarrollo de \textit{Software Libre}, lo cual muestra la gran variedad y libertad existente en el desarrollo de estos proyectos. 

Otra situación curiosa es que la gran mayoría  – el 33\% –  se definen como ingenieros de software y solo el 11\% como programadores, que el 21\% son estudiantes universitarios de carreras afines con IT y un 7\% son profesores de universidades  –  constatando nuevamente el importante papel que juegan las universidades en el mundo del \textit{Software Libre} –  y por último que tan solo el 1\% está integrado en áreas comerciales o marketing.
 
\section{Nacionalidad y residencia de los desarrolladores}

Una de las principales características de los proyectos de \textit{Software Libre} es su metodología descentralizada de trabajo, lo cual permite contar con desarrolladores en cualquier parte del mundo, sin embargo existen zonas geográficas donde la concentración es bastante elevada y donde la ausencia es total. Entre 5478 desarrolladores encuestados se obtuvieron 94 nacionalidades diferentes, de los cuales 705 desarrolladores de 78 países están radicados en otros países. 

Entre las nacionalidades con mayor número de participantes tenemos a Estados Unidos y Alemania encabezando la lista, seguido de Francia, Canadá, Reino Unido y Australia. Cabe destacar la presencia de Brasil en la posición 14 como el país que encabeza la lista entre países sudamericanos.

Es más, si se pudiera contar con un mapa con la ubicación geográfica precisa de cada uno de los desarrolladores de \textit{Software Libre} a nivel mundial, se vería que ese mapa se corresponde a la perfección con los habituales mapas nocturnos en los que se percibe dónde hay luz y dónde no, al presentarse una mayor densidad de desarrolladores en las zonas más desarrolladas del planeta.

A pesar de que la lista por países la encabeza Estados Unidos con un elevado número de participantes, es curioso ver que a nivel de continentes Europa tiene el dominio con un 54.89\% de participantes, y Norte América con un 34.69\%. Por debajo de estos dos tenemos a Oceanía con un 4.9\%, Sur América con un 2.82\%, Asia con un 2.13\% y África con un 0.56\%.

En cuanto a los países de residencia de los desarrolladores, se encontró que muchos de ellos no viven en su país de origen, siendo Reino Unido, Alemania, Francia y Canadá los países con mayor índice de Emigración, y Estados Unidos, Reino Unido, Alemania y Canadá los países con mayor índice de inmigración. 

Es curioso ver la contrariedad que se presenta en esta situación, pues existen países en los que así como hay un alto nivel de emigración también hay un alto nivel de inmigración. Para poder llegar a una mejor conclusión y saber cuáles son los países que más seducen a los desarrolladores para radicarse y cuáles donde quizás no se sienten muy cómodos ejerciendo sus oficios, se realizó una comparación entre los niveles de inmigración y emigración de los diferentes países para poder saber cuáles presentan un flujo migratorio positivo  y cuales un flujo migratorio negativo. El resultado de este análisis muestra a Estados Unidos como el país donde más inmigran los desarrolladores, muy por encima de Suiza, Australia, Japón y Canadá, países que lo preceden. Mientras que Austria, Sur África, Chile y Nueva Zelanda son los países que más nivel migratorio negativo mostraron.

\section{Tiempo de dedicación al \textit{Software Libre}}

Hacer referencia a un tiempo exacto que los desarrolladores de \textit{Software Libre} dedican a la semana a realizar esta actividad no es una tarea fácil y tampoco podemos tomar los resultados de la encuesta como una realidad aproximada; ya que los encuestados al contestar la pregunta sobre el tiempo de dedicación al desarrollo de \textit{Software Libre} pueden interpretarla y considerar la variable tiempo incluyendo o excluyendo diferentes actividades que van desarrollando durante el proceso. Sin embargo, para quienes dedican tiempo al software propietario sí podríamos considerar que los datos son más exactos ya que se cumple un horario establecido o hay una remuneración económica por el desarrollo del software.

Basándonos en los resultados a la pregunta de tiempo dedicado al desarrollo de \textit{Software Libre} tenemos que en total 5233 personas  participaron contestando, de los cuales 1811 contestaron que dedican menos de 5 horas semanales, es decir el 34.6\%; 1654 (31,7\%) dedican entre 5 y 10 horas; entre 10 y 20 horas a la semana 968 (18,5\%); 318 (6\%) dedican entre 20 y horas semanales; 203 (3,9\%) entre 30 y 40 horas; y finalmente, 279 personas (5,3\%) manifestaron que dedican más de 40 horas a la semana al desarrollo de \textit{Software Libre}.

En cuanto al tiempo semanal dedicado al desarrollo de software propietario tenemos: el 18,4\% de los encuestados dedican menos de 5 horas, el 10,7\% entre 5 y 10 horas, el 11,8\% dedica entre 10 y 20 horas, el 37,8\% entre 20 y 40 horas y finalmente el 20,4\% dedica más de 40 horas a la semana.

A modo de conclusión, tenemos que los más altos porcentajes en tiempo dedicado al desarrollo de \textit{Software Libre} se encuentra en menos de 10 horas  semanales con un total del 66,3\% del total de los encuestados y para software propietario el tiempo semanal con mayor dedicación se encuentra entre los rangos de 20 y 40 horas y más de 40 horas con un 37,8\% y 20,4\% respectivamente.

\section{Motivación de los desarrolladores}

Una de las principales preguntas que surgen cuando se habla de \textit{software} libre es \textit{¿Que motiva a los desarrolladores a dedicar tiempo a esto si no reciben retribuciones económicas?}. Esta pregunta genera muchas especulaciones y dudas, pues el concepto generalizado del trabajo en el mundo es que se trabaja principalmente para obtener beneficios económicos. Sin embargo y aunque puede ser considerada una razón, no es el dinero el factor que más motiva a los desarrolladores a dedicar tiempo en proyectos de \textit{Software Libre}.

Antes de analizar los resultados de  las encuestas estudiadas en este capítulo, revisaremos algunos motivos que el \textit{proyecto GNU}\footnote{ http://www.gnu.org/philosophy/fs-motives.html } define dentro de la documentación de su filosofía:

{\bf 1. Fun.} For some people, often the best programmers, writing software is the greatest fun, especially when there is no boss to tell you what to do. Nearly all free software developers share this motive.

\textit{Diversión. Para algunas personas, a menudo los mejores programadores, escribir software es la mayor diversión, especialmente si no hay ningún jefe diciéndole qué debe hacer. Casi todos los programadores de \textit{Software Libre} comparten este motivo.}\vspace{0.4cm} 

{\bf 2. Political idealism.} The desire to build a world of freedom, and help computer users escape from the power of software developers.

\textit{Idealismo político. El deseo de construir un mundo en libertad y ayudar a los usuarios de computadoras a escapar del poder de los desarrolladores de software.}\vspace{0.4cm} 

{\bf 3. To be admired.} If you write a successful, useful free program, the users will admire you. That feels very good.

\textit{Ser admirado. Si escribe un programa útil y de éxito los usuarios le admirarán... y eso sienta bien.}\vspace{0.4cm} 

{\bf 4. Professional reputation.} If you write a successful, useful free program, that will suffice to show you are a good programmer.

\textit{Reputación profesional. Si escribe un programa libre útil y de éxito, será suficiente para demostrar que es un buen programador.}\vspace{0.4cm} 

{\bf 5. Gratitude.} If you have used the community's free programs for years, and it has been important to your work, you feel grateful and indebted to their developers. When you write a program that could be useful to many people, that is your chance to pay it forward.

\textit{Gratitud. Si ha usado \textit{Software Libre} de la comunidad durante años, y han sido importantes para usted, se siente agradecido y en deuda con sus desarrolladores. Cuando escribe un programa que puede ser útil a mucha gente, es su oportunidad de pagar la deuda con la misma moneda.}\vspace{0.4cm} 

{\bf 6. Hatred for Microsoft.}  It is a mistake to focus our criticism narrowly on Microsoft. Indeed, Microsoft is evil, since it makes non-free software. Even worse, it implements Digital Restrictions Management in that software. But many other companies do one or both of these. Nonetheless, it is a fact that many people utterly despise Microsoft, and some contribute to free software based on that feeling.

\textit{Odio a Microsoft. Es un error enfocar nuestras críticas sólo a Microsoft. Ciertamente Microsoft es maligno, dado que hace software que no es libre. Aún peor, implementa la gestión de restricciones digitales [DRM, por sus siglas en inglés] en ese software. Pero muchas otras compañías hacen una de esas cosas, o ambas. Sin embargo, es un hecho que muchas personas desprecian completamente y profundamente a Microsoft, y algunos contribuyen al \textit{Software Libre} basados en este sentimiento.}\vspace{0.4cm} 

{\bf 7. Money.} A considerable number of people are paid to develop free software or have built businesses around it.

\textit{Dinero. A un número considerable de personas se les paga para que desarrollen \textit{Software Libre} o han construido negocios en su ámbito.}\vspace{0.4cm} 

{\bf 8. Wanting a better program to use.} People often work on improvements in programs they use, in order to make them more convenient. (Some commentators recognize no motive other than this, but their picture of human nature is too narrow.)

\textit{Querer usar un programa mejor. Las personas generalmente trabajan para mejorar los programas que usan con el objetivo de hacerlos más convenientes para ellos. (Algunos observadores reconocen éste como el único motivo, pero su percepción de la naturaleza humana es demasiado limitada).}\vspace{0.4cm}


Como puede verse GNU menciona 8 motivos, pero es importante tener en cuenta que la naturaleza humana es compleja y que es muy común que una persona tenga múltiples motivos para decidir sobre una misma acción, o que cada persona encuentre motivos diferentes a los aquí listados. Por esta razón, la suma de los resultados de las encuestas realizadas pueden superar el 100\%, ya que los encuestados tenían la opción de elegir varias opciones.

Basados en la encuesta, puede observarse que el motivo principal para que los desarrolladores dediquen tiempo a trabajar en proyectos de \textit{Software Libre} es porque quieren aumentar y/o compartir sus conocimientos y habilidades como desarrolladores, ya que el 78.9\% respondieron que lo hacen para aprender y desarrollar nuevas habilidades, y un 49.8\% respondieron que lo hacen para compartir sus conocimientos.
El aprendizaje y compartición de conocimiento que se establece tiene una fuerte componente social, entrando en una comunidad independientemente de que los miembros de dicha comunidad en concreto nunca lleguen a conocerse físicamente, y en la que se practican nuevas formas de trabajo colaborativo muy valoradas por las personas que en alguna ocasión las han experimentado.
Por otro lado es interesante ver que tan solo el 4.4\% lo hace por dinero y el 9.1\% por reputación, porcentajes muy bajos teniendo en cuenta que son algunos de los motivos nombrados por la GNU y que más rondan dentro de las teorías clásicas.
Para terminar de desvirtuar la teoría de la reputación se preguntó por el conocimiento sobre varios desarrolladores reconocidos, encontrando que los únicos que realmente tienen un nivel de popularidad bastante alto son los que de cierta forma han realizado grandes aportes al inicio del \textit{Software Libre} (Linus Torvalds y Richard Stallman, con mucha ventaja sobre otras personas importantes dentro del movimiento del \textit{Software Libre} como puedan ser Miguel de Icaza con el proyecto GNOME o Eric S. Raymond, autor de ``The Cathedral & the Bazaar'' y siendo uno de los fundadores de la \textit{Open Source Initiative}, mientras que el resto tienen una popularidad tan baja que su porcentaje compite con el margen de error de la encuesta. De hecho, resulta casi cómico ver cómo algunos de los desarrolladores de las aplicaciones con más descargas en Freshmeat en el momento de la realización de la encuesta, obtienen un índice de popularidad además de bajo (entre el 3\% y el 5\%), muy similar al de otros desarrolladores inexistentes introducidos en la encuesta para controlar el margen de error de la misma. Estos resultados indican por lo tanto que la búsqueda de reconocimiento y de prestigio no es una motivación relevante a la hora de dedicarse al \textit{Software Libre}.
Para terminar, se preguntó a los desarrolladores si han recibido algún beneficio económico con esta actividad, encontrando que a pesar del bajo porcentaje de personas que hacen esto por dinero, más del 43\% ha recibido una remuneración económica indirecta, más del 50\% una remuneración directa y solo el 46.3\% no ha recibido ningún tipo de remuneración.

\section{Mitos y realidades}

Una de las creencias más comunes cuando se habla de desarrolladores de \textit{Software Libre}, es que se trata de personas en su gran mayoría varones, cuya afición por la informática y la tecnología es lo único que les importa en su vida, se cree que son personas asociales, solitarias y enfrascadas en su propio mundo.

Después de realizar todos estos análisis provenientes de cifras reales y considerando algunas encuestas que mencionaremos a continuación, podemos afirmar que de esta creencia una parte es realidad y otra parte no es más que mito.

En primer lugar debemos considerar totalmente cierta la creencia de que la mayoría de los desarrolladores son hombres. Según varias encuestas, el desarrollo de \textit{Software Libre} predomina en el género masculino y tan sólo un pequeño porcentaje que varía entre el 1\% al 3\% son mujeres, valor que puede ser aún menor teniendo en cuenta que se encuentra por debajo del margen de error de 3.5\% de las encuestas.

Por otro lado podemos afirmar que la creencia de que estas personas son seres asociales y solitarios no es cierta, pues no concuerda con lo demostrado en estos estudios. Primero que todo podemos ver que los desarrolladores son seres comúnes y corrientes, de diferentes edades, universitarios, profesionales, con diferentes ocupaciones, aficiones, etc. Además, y para terminar de desvirtuar esta creencia, consideraremos una última encuesta donde el 60\% de los desarrolladores afirmó tener pareja y el 16\% afirmó tener hijos.
