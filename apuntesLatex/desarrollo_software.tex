\chapter{Desarrollo de software libre. Organizaci\'on de las comunidades}
\label{CHAP5:Development}
\section{Introducci\'on}
Las reglas de gobierno de las comunidades son muy diversas. Las normas en el software libre tienen que ser muy inclusivas, es decir tienen que estar de acuerdo, para que las personas sigan en la comunidad contribuyendo.

El factor primero es el tama\~no, ya que si es peque\~no normalmente no tiene organizaci\'on, pero si es muy grande seguramente tiene una organizaci\'on. Cuando una persona quiere contribuir tiene que ser consciente de la organizaci\'on. Las normas pueden aparecer o no escritas en la documentaci\'on de la organizaci\'on. Si uno es desarrollador debe de cumplir las normas del proyecto. Todo esto se aplica a la econom\'ia de atenci\'on.

Vamos a estudiar las organizaciones desde el punto de vista de la organizaci\'on del proyecto. Seguidamente se ver\'a como son las condiciones de contribuci\'on y como se fomentan las contribuciones externas.

A continuaci\'on vamos a estudiar algunos proyectos.

\section{Proyecto Apache (ASF)}
\section{OpenOffice}
\section{Mozilla}