% Definimos el estilo del documento
\documentclass[oneside,12pt,a4paper,spanish]{book}
% Utilizamos el paquete para utilizar español
\usepackage[T1]{fontenc}
\usepackage[spanish]{babel}
% Utilizamos el paquete para gestionar acentos
\usepackage[utf8]{inputenc}
%Utilizamos el paquete para incorporar graficos postcript
\usepackage[dvips,final]{epsfig}

\usepackage[spanish]{babel}
\usepackage{graphicx}
\usepackage{a4wide}
\usepackage{hyperref}

\usepackage[normalsize]{subfigure}
\usepackage{color}


%Para que tenga los saltos de linea y no este todo tan junto
\usepackage{geometry}
\geometry{a4paper, left=2.6cm, right=2.2cm, top=3.0cm, bottom=3.0cm}
\setlength{\parskip}{1ex plus 0.5ex minus 0.2ex}

\DeclareGraphicsExtensions{.png,.jpg,.pdf,.eps,.sxd,.odg}

\pretolerance=10000
%Para que no corte las palabras al final de la linea.

% Definimos la zona de la pagina ocupada por el texto

%Empieza el documento
\begin{document}
% Definimos titulo, autor, fecha, generamos titulo e indice de contenidos 
\thispagestyle{empty}

\vspace{2cm}
\begin{figure}[htb]
\centerline{\resizebox{.60\textwidth}{!}{\includegraphics{img/logo_urjc}}}
\end{figure}


\begin{center}
{\Large {\bf Máster en Sistemas Telemáticos e Informáticos}}
\vspace{5mm}

{\large Escuela Superior de Ciencias Experimentales y Tecnología}
\vspace{5mm}

{\large Curso académico 2010-2011}

\vspace{2cm}

{\Huge {Apuntes Software Libre }}

\end{center}


% License
\thispagestyle{empty}
\begin{flushright}
	\footnotesize{\ \ \ Copyright
	\copyright~2011 \textit{MSTI-SWL-Group} \\
	Este documento se publica bajo la licencia \\
\href{http://creativecommons.org/licenses/by-sa/3.0/es/}{Creative Commons
Reconocimiento-Compartir bajo la misma licencia 3.0 España}.
http://creativecommons.org/licenses/by-sa/3.0/es/}
\end{flushright}
\newpage
\thispagestyle{empty} \cleardoublepage

%Indice
\tableofcontents
%Indice de figuras
\listoffigures

%Capitulos
\chapter{Introducción al \textit{Software Libre}}
\section{Introducción y Motivación al \textit{Software Libre}}

TODO: by jfernandez

\section{Historia del \textit{Software Libre}}

TODO: by jfernandez

\section{La Definición de \textit{Open Source}}

\subsection{Motivación}

Existe un motivo fundamental para definir aquello de lo que estamos hablando:
evitar confusiones. Por ello, en esta sección estableceremos a qué nos estamos
refiriendo al hacer mención de aspectos como \textit{Free Software},
\textit{Open Source} o \textit{Software Libre}.

Para ello, debemos recurrir en primera instancia a Richard Stallman y la
\textit{Free
Software Definition}\footnote{http://www.gnu.org/philosophy/free-sw.html}, cuya
lectura detenida queda totalmente recomendada. En esta definición, se establecen
las cuatro libertades que debe ofrecer un \textit{software} para que pueda ser
considerado como libre, y que se detallan a continuación:

\begin{enumerate}
  \setcounter{enumi}{-1}
  \item La libertad de ejecutar el programa para cualquier propósito. De lo
cual se desprende que no se debería prohibir el uso del programa, por ejemplo,
con fines militares, si bien es cierto que esto no ampara los posibles usos para
fines ilegales.
  \item La libertad para estudiar cómo funciona el programa y modificarlo como
se desee. De esto se desprende lo que se podría considerar como un corolario:
es necesario tener acceso al código fuente.
  \item La libertad para redistribuir copias ya que con ello, puedes ayudar a
tu vecino. La manera de redactar esta libertad en concreto, deja entrever la
filosofía de vida y pensamiento de Richard Stallman.
  \item La libertad para redistribuir copias del \textit{software} una vez
modificado. Se puede entender como una combinación de las dos libertades
anteriores.
\end{enumerate}

Como análisis de estas libertades, se puede afirmar que Richard Stallman en el
momento de su definición, estaba poniendo el énfasis en la idea de la libertad
(en el sentido de la libertad de expresión). De hecho, en ningún momento se
recogen menciones al ámbito comercial y/o económico, ya que para tener libertad
de expresión no es necesario pagar (obsérvese la desambiguación del término
inglés \textit{``free''}, indicándose que debe entenderse en el sentido de
\textit{``free speech''} y nunca en el sentido de \textit{``free beer''}, por lo
que igualmente se hace énfasis de una manera indirecta en que por el hecho de
ser libre, el \textit{software} no tiene por qué ser gratuito).

En cualquier caso, la \textit{Free Software Definition} es tan sólo una
definición que nos ayuda a identificar y concretar a qué nos referimos al
emplear el término \textit{Free Software}. Pero a la hora de materializar este
concepto en la práctica, ¿cómo ha de hacerse? Debido a la propiedad intelectual
y sus mecanismos, si no se da permiso explícito, esa obra pertenece a su creador
y nadie más puede hacer uso de ella.

De aquí nace la necesidad de contar con el concepto de licencia, mecanismo
empleado para indicar al usuario qué puede hacer y qué no con ese
\textit{software}, y que se estudiará en detalle en el capítulo
\ref{CHAP2:Licenses} de este documento. De este modo, será necesario
leer la licencia de ese \textit{software} y en caso de cumplirse con las cuatro
libertades indicadas anteriormente, nos encontraremos ante un \textit{software}
libre. De lo contrario (bastará con que no se cumpla tan sólo una de las cuatro
libertades) estaremos ante un \textit{software} privativo o \textit{software}
propietario (estos dos últimos términos se emplean como sinónimos, pese a que lo
correcto es el término privativo, ya que por el hecho de hacer que un
\textit{software} sea libre, no se pierde la autoría desde el punto de vista
legal, se continúa siendo el propietario de cara a la ley).

En cualquier caso, resulta indudable que la lectura y análisis de una licencia
\textit{software} es un trabajo tedioso. Por ello, es preferible contar con un
listado de licencias asimiladas y reconocidas como de \textit{software} libre.
Además, este listado de licencias deberá ser emitido por una entidad de
reconocido prestigio en la materia.

Retomando la definición ofrecida por la \textit{Free Software Foundation}, debe
notarse que se trata de un lenguaje abstracto y de altísimo nivel, mientras que
una licencia emplea un lenguaje muy concreto de lo que se puede y lo que no se
puede hacer. Este fue el motivo por el que nacieron las \textit{Debian Free
Software Guidelines}\footnote{http://www.debian.org/social_contract#guidelines}
(lectura totalmente recomendada): dado que la definición original es de muy alto
nivel, se deben establecer reglas más detalladas, de tal modo que se pase de
hablar de aspectos como la libertad de expresión, a cosas mucho más concretas.

Este trabajo llevado a cabo por \textit{Debian} tuvo gran éxito y buena acogida,
y por ello fue tomado como base para otras definiciones, como es el caso de
la \textit{Open Source Definition}. Pero, ¿por qué es necesaria una nueva
definición para una misma cosa? La razón de esto se encuentra en que pese a
todas las matizaciones e indicaciones, \textit{Free Software} puede referirse en
inglés tanto a \textit{``libre''} como a \textit{``gratis''}. Por ello comenzó
también a emplearse el término \textit{Open Source}, dándose origen a la
\textit{Open Source Definition}, que vamos a estudiar en detalle a continuación.


% Cap�tulo incluido en los Apuntes Colaborativos realizados por los
% alumnos de Software Libre en el M�ster Universitario Oficial en
% Sistemas Telem�ticos e Inform�ticos (Universidad Rey Juan Carlos),
% durante el curso 2010/11.

\chapter{Aspectos Legales}
En este cap�tulo se ver�n los elementos principales involucrados en
los aspectos legales del software libre. El m�s importante de ellos es
la licencia, que supone el contrato establecido entre el autor y los
usuarios y que define qu� libertades tienen estos �ltimos sobre la
obra en cuesti�n. Antes de entrar en detalle con las licencias se dar�
un barrido sobre la propiedad intelectual, que engloba conceptos como
\emph{Copyright} y patente, entre otros.

\section{La propiedad intelectual}
La propiedad intelectual engloba los derechos de autor, las patentes,
el secreto industrial y las marcas. Es compleja, principalmente por la
dificultad de marcar sus l�mites. Prueba de ello es el caso del puente
Zubi~Zuri bilba�no, en el que Santiago Calatrava, dise�ador del
puente, denunciaba las modificaciones que el ayuntamiento de la ciudad
efectu� sobre su obra. En una primera instancia, la reclamaci�n del
pago de una indemnizaci�n de 250.000� fue denegada. Tras la
presentaci�n de un recurso, el arquitecto recibi� 30.000�. Esta
sentencia deja la sensaci�n de que el bien p�blico se antepone sobre
la propiedad intelectual. Hay otros casos interesantes como la demanda
que el artista Nach interpuso sobre el Ministerio~de~Sanidad, motivado
por un anuncio que fomentaba el uso del preservativo. Tanto el tema
``Efectos~Visuales'' del cantante como la canci�n del anuncio
compart�an estilo m�sical y versos en los que se utilizaba una �nica
vocal. La petici�n fue desestimada. El omnipresente Google tambi�n ha
sufrido demandas de este tipo. Su programa de recopilaci�n de noticias
Google~News hac�a que los peri�dicos autores de las noticias perdieran
ingresos al disminuir los interesados en publicitarse en sus p�ginas
oficiales. Como se puede observar, la propiedad intelectual afecta a
muchos sectores. En las siguientes secciones veremos c�mo nace el
concepto, las diversas interpretaciones geogr�ficas y las pr�cticas
habituales para salvaguardarla.

\subsection{Or�genes de la propiedad intelectual}
La propiedad intelectual nace junto con uno de los mayores inventos de
la historia: la imprenta. Hasta ese momento, la copia manual era la
�nica forma de reproducci�n de obras p�sible. Los autores ni siquiera
se hab�an planteado el problema que vendr�a a�os despu�s en
Inglaterra.

La imprenta ofrec�a la posibilidad de reproducir libros con gran
rapidez, lo que desemboc� en nuevos modelos de negocio. La venta de
libros y peri�dicos a gran escala se convert�a as� en algo
factible. Los due�os de las imprentas empezaron a rentabilizar sus
m�quinas. Los autores, conscientes del nuevo mercado, exig�an su parte
de beneficios. Para ello ten�an que llegar a acuerdos con los
impresores. Pero, \emph{�qu� pasaba si otra imprenta se hac�a con una
  copia y la reproduc�a sin llegar a un acuerdo con el autor?} Para
solventar ese problema, los ingleses realizaron ``pactos de
caballeros'' mediante los que no se publicar�a ninguna obra sin el
consentimiento del escritor. Pronto llegar�an los galeses para
salt�rselos.

Los empresarios ingleses empezaron a manifestar su animadversi�n
contra sus competidores galeses. Ellos no pagaban a los autores, por
lo que su margen de beneficios era mayor. Para hacer frente al
problema, acudieron a la Reina~Ana~de~Gran~Breta�a, m�xima autoridad
de la �poca, y expresaron sus quejas. La Reina encontr� una soluci�n
para los impresores ingleses y sobre todo para s� misma: impuso un
permiso de copia que s�lo pod�a ser otorgado por la Corona. Con esta
decisi�n adquiere un control total sobre los libros que se imprim�an,
que podr�an contener ideas subversivas contra el reino. Nace as� la
censura en las imprentas.

M�s tarde, surge en Francia una nueva rama de la propiedad
intelectual. Consiste en la protecci�n del autor por encima del resto
de factores. Supone que el artista es tocado por Dios, lo que le
permite crear su obra. Esta rama ha tenido una gran influencia en
nuestra legislaci�n actual\footnote{En palabras de Grex: ``�nuestros
  artistas est�n afrancesados!''}, en la que se separa
\emph{propiedad~intelectual} de \emph{propiedad~industrial}. Una vez
introducido el origen de las pr�cticas, pasamos a ver uno de los
elementos que m�s perjudica al Software Libre, las patentes.

\subsection{Las patentes}
Una patente otorga el monopolio del invento a su creador durante un
periodo que va desde 17 a 25 a�os, seg�n la regi�n. El inventor ha
dedicado un tiempo al trabajo y recibe esta recompensa. A cambio, el
invento debe ser p�blicado. Las terceras partes que quieran trabajar
sobre �l deben llegar a un acuerdo con el titular de la patente.

No se puede patentar una idea. La instituci�n registradora de patentes
se encarga de acotar los l�mites del invento. \emph{La patente de una
  silla, �deber�a permitirme explotar tambi�n sillones y sofas?} Sin
embargo, hay organizaciones como la \emph{Organizaci�n Mundial de la
  Propiedad Intelectual} (OMPI) que est�n ejerciendo presi�n para que
se relajen las restricciones de registro de una patente. No es un
procedimiento al alcance de cualquiera por lo que unas pocas empresas
suelen ser due�as de la inmensa mayor�a de patentes. Por otro lado, se
ha detectado una pr�ctica denominada \emph{patent trolling} que
engloba a las entidades que se encargan de patentar el mayor n�mero
posible de inventos, no para explotarlos y lucrarse con ellos, sino
para obtener beneficios mediante los acuerdos con interesados en el
invento.

En el software libre las patentes suponen un problema muy grande. El
acceso al c�digo fuente facilita la detecci�n de infracciones sobre
patentes. Los autores de software privativo tienen m�s ventajas en
este sentido. En Europa, a diferencia de Estados Unidos, todav�a no es
posible patentar software, aunque podr�a llegar ese momento.

\subsection{El secreto comercial}
Las empresas tienen el derecho a no hacer p�blico su trabajo. Un
ejemplo muy claro de este tipo de pr�ctica es la ocultaci�n de la
f�rmula de la \emph{Coca-Cola}. A la compa��a no le interesa una
patente, ya que saben que pueden rentabilizar su producto mucho m�s
all� de la vigencia de �sta. Otro ejemplo es la ocultaci�n de la
ingenier�a utilizada entre las diferentes escuder�as de la
\emph{F�rmula 1}, donde el prestigio por ganar el campeonato tiene un
peso muy grande. En ambos casos se pretende esconder los secretos de
la empresa para dificultar el trabajo a los competidores.

El secreto comercial resulta m�s limpio que una patente, en t�rminos
de mercado. La raz�n es que permite la coexistencia con terceras
partes que se dediquen a imitar el producto original. No hay m�s que
ver las infinitas variedades de refrescos de cola que se pueden
adquirir, habi�ndose convertido \emph{Pepsi} en el adversario m�s
fuerte. En la imitaci�n tiene una gran importancia la ingenier�a
inversa, que permite deducir el proceso de manufacturaci�n a partir
del producto final. Los secretos mejor guardados se encuentran en las
grandes compa��as, que como viene siendo habitual, ejercen mucha
presi�n sobre el Estado cuando se ven amenazados. Por eso, existen
regiones en las que la ingenier�a inversa est� prohibida.

En el software, el proceso de creaci�n de un producto es encarnado por
el c�digo fuente. Lo que permite el secreto comercial es la
distribuci�n del binario. Por lo tanto, est� pr�ctica carece de
sentido en el software~libre, ya que aplicarla ir�a en contra de las
libertades b�sicas al no disponerse del fuente.

\subsection{Las marcas}
Una marca es el nombre ---y opcionalmente el logotipo--- que utiliza
un negocio para promocionarse. Registrar una marca requiere cierto
desembolso econ�mico, por lo que generalmente no suele
realizarse. Esto explica en cierta manera por qu� es posible que
existan cientos de bares ``Pepe'', pero dificilmente se encontrar� una
tienda deportiva que utilice el nombre ``Nike''.

En el software~libre, son s�lo los grandes productos y compa��as los
que las utilizan. Entre ellos encontramos a Debian, GNOME o GNU. En la
historia del software~libre se han encontrado problemas por el
no~registro de una marca. El caso m�s representativo es el de Linux,
en el que varias personas registraron esa marca y reclamaron royalties
a las distintas distribuciones existentes. Tambi�n es un problema de
marcas el que tuvieron Firefox y Debian. Esta distribuci�n se neg� a
incluir el navegador por la falta de libertad sobre los logos
oficiales y el nombre de Firefox. La soluci�n fue incluir el navegador
con el nombre y el logotipo renombrados, dando como resultado el
proyecto IceWeasel.

\subsection{Derechos de autor}
Los derechos de autor surgen para recompensar a los autores de libros
o de arte. Representan lo que tambi�n es conocido como
\emph{Copyright} y tratan de proteger la expresi�n de un contenido, no
el contenido en s� mismo. Por dar un ejemplo actual, todos los d�as
vemos en los peri�dicos art�culos que tratan sobre los mismos sucesos,
pero contados de forma diferente. Si se protegiese el contenido, la
noticia s�lo podr�a ser publicada en un peri�dico.

En Espa�a, la encargada de estos derechos~de~autor es la denominada
Ley~de~Propiedad~Intelectual. Esta ley hace una subdivisi�n en
Derechos~Morales y Derechos~Patrimoniales. Los primeros hacen posible
la autor�a de una obra y que se respete su integridad. Son vitalicios
o indefinidos. Los segundos son los que permiten al autor la
explotaci�n econ�mica de su trabajo. Tienen un periodo de vigencia de
70 a�os\footnote{Otro ejemplo de presi�n sobre la legislaci�n de la
  propiedad intelectual es la que ejerce
  Mickey~Mouse. Misteriosamente, la ley se prolonga cuando este
  simp�tico rat�n est� a punto de pasar al dominio
  p�blico.}. Actualmente, la forma de obtener los derechos~de~autor es
autom�tica. En cuanto se crea una obra, entra en vigor. El autor de un
garabato en una pizarra dispone de todos los derechos sobre �l. Hace
a�os no se aplicaban por defecto. Es curioso el caso de la m�tica
pel�cula de t�tulo \emph{La noche de los muertos vivientes} en la que
olvidaron incluir la (C) de Copyright, por lo que directamente pas� a
formar parte del dominio p�blico.

\subsection{�Qu� pasa con el Software?}
\label{sub:que_pasa}
Hasta el momento hemos visto diversos mecanismos para tratar la
propiedad~intelectual. \emph{�En cual de ellos encaja el Software?} En
un principio hubo mucha pol�mica por decidir si se deb�an aplicar
patentes o derechos de autor, finalmente se impuso el segundo de
ellos. No resultar�a mantenible solicitar una patente por cada commit
realizado. Adem�s exist�an algunas similitudes con las novelas y dicha
legislaci�n hab�a funcionado bien hasta el momento. No tardar�an en
aparecer incongruencias y situaciones especiales. Y es que el software
tiene unas caracter�sticas que le hacen muy diferente del resto de
obras existentes hasta la fecha.

\section{Licencias}
Si navegando por \emph{la Red} descubrimos un software sin licencia,
\emph{�qu� podr�amos hacer con �l?} Acabamos de ver en
\ref{sub:que_pasa} que lo que aplicamos al software son los
derechos~de~autor, es decir, la obra autom�ticamente pasa por defecto
a restringir todos los derechos de los usuarios. Por tanto, si no
encontramos la licencia en dicho programa, debemos asumir que no
podemos hacer absolutamente nada con �l. Bien es cierto que ser�a
dif�cil que un juez declarara culpable a una persona por usar ese
software. Sin embargo, la modificaci�n y redistribuci�n ya s� que
ser�a un terreno muy pantanoso.

Entonces, \emph{�qu� es una licencia?} Es un contrato mediante el cual
el autor se comunica con los usuarios finales. Aunque esto a veces es
dif�cil de entender, la licencia no restringe, sino que se encarga de
otorgar libertades. Sin ella, comprar un programa no nos dar�a la
capacidad de usarlo tan siquiera, se debe recordar lo que se coment�
en el p�rrafo anterior. La licencia es un contrato muy especial, un
contrato que ni firmamos ni aceptamos expl�citamente. En ocasiones el
instalador pide la confirmaci�n mediante la t�pica checkbox
\emph{I~Agree}, pero hay casos en los que el entorno no permite esta
posibilidad, como por ejemplo el software embebido en una lavadora.

Las licencias son textos largos y complicados. \emph{�Qu� pasa si no
  entendemos alguna en concreto?} Siempre que se produce un acuerdo,
ambas partes deben ser conscientes de las implicaciones que
conlleva. Por esa raz�n los bancos encuentran muchos problemas con sus
clientes, hasta tal punto que algunos acuerdos requieren que el
interesado tenga un cierto nivel educacional. En el caso del software
la soluci�n es m�s sencilla. Si no entendemos la licencia, se debe
suponer la establecida por defecto, con lo que no se tendr�a derecho a
nada. Otro aspecto habitual es el de encontrar una licencia en otra
lengua. Por ejemplo, la FSF s�lo considera como v�lida la GPL en
ingl�s. Las traducciones pueden dar lugar a segundas interpretaciones,
a veces incluso provocadas. Ante este caso, siempre hay que quedarse
con la interpretaci�n m�s restrictiva o consultar directamente al
autor si fuese posible.

Licenciar software con una licencia libre b�sica es sencillo. Tan s�lo
debemos incluir las siguiente l�neas al principio de los ficheros
fuente del programa:
\begin{quotation}
Copyright (c) 2010 Foobar Developers. All rights reserved.

Redistribution and use in source and binary forms, with or without
modification, are permitted provided that the redistributions of
source code must retain the above copyright notice.
\end{quotation}
Con la primera l�nea se est� ``marcando el terreno''. En ella se puede
apreciar el a�o de creaci�n, el nombre del programa y el nombre de los
autores. La segunda parte del texto es la que otorga al usuario final
la libertad de uso, redistribuci�n, modificaci�n y redistribuci�n de
la modificaci�n (principios b�sicos del software libre). Resulta
curioso observar como primero se reservan todos los derechos para
posteriormente concederlos. La �nica restricci�n que se impone al
usuario ---en este ejemplo particular--- es la de mantener la primera
l�nea tal cual, con el fin de preservar la autor�a. En los siguientes
apartados se describir�n las distintas familias de licencias,
nombrando para cada una sus ejemplares m�s representativos.

\subsection{Licencias Acad�micas}
Son tambi�n conocidas como licencias Minimalistas o de tipo BSD. Son
muy adecuadas para investigaciones acad�micas, generalmente
financiadas por el Estado. Pretenden tener un alcance global, por lo
que no desean cerrar ninguna puerta. Esto conlleva otorgar todas las
libertades al usuario final. La �nica condici�n a cambio es la de
conservar la autor�a de los creadores. Un ejemplo de software en el
que se aplica esta licencia es el protocolo TCP/IP. Es un protocolo
utilizado por cualquier Sistema~Operativo moderno, tanto libre como
privado. Las libertades que ofrece esta familia rozan las que podemos
encontrar en un software de dominio~p�blico.
\begin{description}
\item[Licencia BSD] Es la m�s representativa de esta familia. No
  ofrecen ninguna garant�a al usuario final. �ste es una aspecto que
  siempre ha generado cierta controversia, aunque es bastante
  comprensible\footnote{Los desarrolladores de software~propietario
    tan s�lo garantizan que el soporte es correcto y que el programa
    se ejecuta.}. La permisividad de esta licencia provoca que el
  software se pueda redistribuir tanto en versi�n de c�digo fuente
  como en binario. Para el primer caso, se debe mantener la l�nea del
  Copyright y la lista de condiciones en los ficheros fuente. Con el
  formato binario se debe reproducir de alguna manera el mismo texto
  que para el caso anterior. Por ejemplo, Windows muestra la nota BSD
  durante la carga del sistema. Se puede leer la plantilla de esta
  licencia en
  \url{http://www.opensource.org/licenses/bsd-license.php}.
\end{description}

\subsection{Licencias Permisivas}

% FIXME: Verificar el siguiente p�rrafo, �es una deducci�n personal!
% En las transparencias de licencias vistas en clase se diferencian
% licencias acad�micas de licencias permisivas. Sin embargo, en el
% libro de la asignatura se muestran como un solo conjunto.  

Las licencias permisivas son muy similares a las licencias acad�micas,
hasta tal punto que podr�amos ver a estas �ltimas como un subconjunto
de esta nueva familia.

\subsection{Licencias Copyleft}

\subsubsection{Copyleft d�bil}

\subsubsection{Copyleft fuerte}



\chapter{El desarrollador y sus motivaciones}
\label{CHAP3:Developer}

\chapter{Aspectos económicos y de negocio}
\chapter{Modelos de Negocio}
\label{CHAP3:Economy}

En este capítulo la idea fundamental que se va a presentar es ¿Quién paga el
desarrollo de los programas?, ¿Por qué alguien paga por un software?, ¿Cómo
financiar proyectos y recursos?
% <+> 
Dejaremos de un lado los aspectos éticos y nos centraremos en los
modelos prácticos que permiten obtener beneficios, no siempre
económicos.
% </+>

La financiación de los proyectos pueden llegar de diferentes maneras:

\begin{itemize}
  \item Financiación externa. Alguien financia el proyecto y decide cómo y en
  qué se gastan los recursos.
  \item Autofinanciados. La financiación proviene de las actividades de la
  empresa.
  \item Desarrollos con financiación indirecta. Participación de desarrolladores
  en proyectos pero que trabajan para otras empresas.
  \item Desarrollo para uso interno.
  \item Modelos mixtos. Una pequeña mezcla de los anteriores.
\end{itemize}

A continuación vamos a comentar más en detalle cada uno de los tipos.


\section{Financiación externa}

Como bien indica el título de esta sección, la financiación proviene de una fuente externa
que por diversos motivos quiere que el software sea libre. En los siguientes
apartados vamos a poder ver las diferentes formas de financiación externa que
existen.

\subsection{Empresa pública}


La empresas públicas tiene diversos modelos de financiación, uno de ellos es
del tipo I+D y se decide que para poder financiar un proyecto, éste tendrá que
ser libre.

Otro modelo de financiación es crear proyectos competitivos, como por ejemplo
con los planes Avanza, que se evalúan según lo que pueden ofrecer a la
investigación. Éstos pueden ser software libre.
% <+> 
También hay cabida para motivación precompetitiva, donde se
pretende desarrollar una tecnología que pueda ser explotada en el
futuro por determinadas empresas. El acceso público al programa
permite que el rango de grupos que pueden hacer uso de él sea mucho
más grande, favoreciendo la entrada de PYMES y modestos proyectos de
investigación.
% </+>

Muchas veces con este tipo de financiación no se busca recuperar la inversión,
aunque hay casos en los que se puede recuperar mediante subproductos, pero no
es la idea.
% <+>
Un ejemplo de esto puede ser el provecho que la sociedad obtiene por
la inversión en tecnologías sanitarias para un hospital público.
% </+>

Las motivaciones para la financiación de este tipo de proyectos son variadas,
como científicas, precompetitivas, promocionales, sociales o políticas.
% <+> 
En las científicas podemos encontrar, por ejemplo, estudios en
biotecnología, donde los resultados pueden ser reutilizados por otras
entidades. En cuanto a las precompetitivas, se busca que todo el tejido
industrial se pueda beneficiar de esos resultados precompetitivos, siendo el
software libre en este caso, tal y como se ha podido comprobar, un claro
vehículo para conseguirlo, ya que al finalizar la producción de ese software,
se libera (téngase en cuenta que se subvenciona el software no sólo para su
producción sino para su posterior liberación) quedando a disposición de
todos, incluso para aquellos que no pertenecen al consorcio del proyecto. Las
promocionales se basan generalmente en extender un estándar, donde el ejemplo
más ilustre es el de Internet. Gracias a los reducidos costes que suponía
implantar TCP, fueron muchos los fabricantes que se lanzaron a incluirlo en sus
productos. En cuanto a la motivación social, se hace presente en ciertas
ocasiones cuando sale más rentable crear un producto desde cero y publicarlo
como Software Libre que financiar otro producto de pago ya existente. Un
ejemplo de esto sería imaginar una sociedad en la que no existiesen
navegadores web gratuítos.
% </+>

Un caso de estudio importante en esta sección es el del compilador de lenguaje
Ada llamado GNAT. Su principal objetivo era conseguir un compilador muy barato,
pero su calidad y éxito fue tal que llegó a expulsar del mercado a otros
compiladores de Ada privativos. La clave de su éxito (y de su coste
reducido) se encontró en que no se realizó un desarrollo del mismo desde cero,
sino que se reutilizaron otros programas de software libre como el compilador
de C, GCC. En definitiva, el esfuerzo se centró en la creación del front-end
para la sintaxis del lenguaje, dado que el resto de componentes ya estaban
hechos.

Dentro de la financiación externa, existen algunas variantes de las cuales
hablaremos a continuación.

\subsection{Empresa privada sin ánimo de lucro}

Normalmente esta financiación está realizada por fundaciones u ONGs, ya bien
sea para producir software o para solucionar problemas mediante el desarrollo de
software.

% <+>
Los motivos de la entidad financiadora suelen ser directos, como el
caso de la FSF cuya meta es la proliferación del software libre, o
indirectos como la fundación Open Bioinformatics, que se vale del
software libre como medio para extender sus productos. También se
pueden encontrar aquí ejemplos de tecnología de geolocalización, para
desplegar las posiciones de víctimas, puestos sanitarios o puntos de
repartición de alimentos tras producirse desastres naturales.
% </+>

La forma de financiar los proyectos es muy parecida a la que se usa en las
empresas públicas.


\subsection{Porque alguien necesita mejoras}
En este caso sí existe el ánimo de lucro. El proyecto en cuestión se financia
porque va a realizar mejoras a un software que necesito.
Esta financiación se puede hacer de varias maneras:
\begin{itemize}
  \item Pagando a una empresa externa.
  \item Pagando a una fundación.
  \item Contratando personal.
\end{itemize}

De cualquiera de las tres maneras se estaría financiando el proyecto del cual
finalmente saldremos beneficiados.


Uno de los casos es \emph{Corel}, empresa que quería portar todo su software a Linux para
poder competir con Microsoft. Y lo que decidieron es usar un emulador llamado
\emph{Wine}, y lo financiaron para que sus aplicaciones funcionaran bien en
Linux, y pagaron a una empresa llamada \emph{Macadamian} para que lo realizase.

\subsection{Indirecta}

La idea de la financiación indirecta es que financio un proyecto, no tanto
porque me interese el proyecto en sí, sino por los productos que puedan salir de
él. En el caso de Google, por ejemplo, que financia un proyecto de software
libre y obtiene un producto como Android.

Google se introduce en el mercado móvil para poder tener control y estar de
intermediarios de todo lo posible como publicidad, aplicaciones, etc\ldots
Crean un sistema operativo como Android, no para ganar dinero con Android, pero
si para no dejar que otros le quiten el poder de ser intermediarios y ser competitivos.
Buscan beneficios en productos relacionados con Android, no del propio Android.
Y el software libre es el instrumento para hacerlo. Intentar obtener céntimos
por cada compra/venta en el mundo con sus productos. La ganancia de Android, es
que lo pueda usar mucha gente y cuanta más gente mejor, bien en dinero o bien en
control.

En el caso de VISA/MASTERCARD hacen algo de este estilo, y la idea de Apple o
Google es ésta. RIM está creando WebOS con software libre pero es privativo.

Tenemos varios ejemplos:
\begin{itemize}
  % <->
  %\item Con Perl, crearon el software y se recuperó el dinero vendiendo libros.
  %Cuantos más usuarios de Perl más gente compraría el libro. Y en vez de
  %pagar por crear el libro, se pagó porque Perl fuese mejor y tuviese más
  %versiones.
  % </->
  % <+>
  \item En ocasiones, una entidad invierte en cierta tecnología con el
    fin de obtener más ventas en productos relacionados donde tienen
    una alta cuota de mercado. Esta estrategia ha sido realizada en
    varias ocasiones por la editorial O'Reilly. El caso más conocido
    es el de su guía sobre el lenguaje de programación Perl. Su libro
    se convirtió en el mayor referente sobre el tema. Por ello
    a la editorial le resultaba rentable invertir capital en el
    desarrollo de este lenguaje. Bien es cierto que terceras partes
    también se lucraban sin gasto alguno con las inversiones
    realizadas por O'Reilly, pero era un factor que no les preocupaba,
    siempre y cuando no cesase su propio crecimiento.
  % </+>
  \item Con el Hardware, VA (VA-Linux) vendía servidores, y lo que hace ahora es vender
servidores con Linux preinstalado y pagó para que las versiones de Linux
de sus servidores corriesen muy bien y sin problemas. 
% <+>
Es cierto que no existía mucho mercado para máquinas con Linux
preinstalado, pero tener una cuota del 80\% de dichas ventas suponía
unos ingresos elevados.
% </+>
Hay casos en los que se
pagan para que hagan el driver para ciertas tarjetas y cuando están disponibles
para Linux su venta aumenta.
\item Otro caso son las distribuciones de Linux, como Ubuntu, que paga para
que la distribución sea conocida y mucha gente la utilice.
% <+>
Ubuntu invierte una gran cantidad en lograr una buena usabilidad y
mejorar la experiencia de usuario. Otras distribuciones como Red Hat
utilizan las certificaciones y consultorías como principal fuente de
ingresos. En ambos casos, lograr que el nombre de la empresa sea
conocido es de vital importancia.
% </+>
\end{itemize}

% <+>
\subsection{Otros modos}
Obtener los ingresos mediante otros mecanismos ajenos a la venta de
licencias, tal y como ocurría en el modelo tradicional, ha llevado a
modelos muy dispares e imaginativos de financiación. Por citar alguno,
podríamos hablar de mercados puntos de encuentro entre desarrolladores
y clientes, venta de bonos para financiación (utilizado por el
proyecto Diaspora), cooperativas de desarrolladores y sistemas de
donaciones (de manera directa realizando una transferencia o con
cierta indirección, como la existente en el market de Android, donde
puedes elegir obtener la misma aplicación sin coste alguno o
pagando un precio razonable a modo de donación).
% </+>

\section{Autofinanciados}

Con la autofinanciación soy yo el que invierte en el desarrollo y luego
trato de recuperarlo de alguna manera.

Estos casos se darían cuando por ejemplo, una empresa quisiera iniciar un
proyecto libre destinado a un determinado nicho de mercado en el que ve
posibilidades de negocio futuro y por tanto, con la idea de rentabilizar a
posteriori esa inversión inicial.

Como se puede entender, mediante este tipo de financiación se pueden originar
diversos modelos de negocio que veremos en este apartado.

\subsection{Mejor conocimiento}

La idea es tratar de vender que tengo el mejor conocimiento de un producto, esto
puede hacer que después cuando haya que realizar un cambio pueda ofrecer precios
más baratos que si lo hace otro porque conozco muy bien el producto o bien para
intentar obtener una marca. Esto se puede dar o bien desarrollando el propio
programa o bien trabajando en otro software que ya existe.
% <+>
La manera más habitual de vender un buen conocimiento es mediante la
inclusión de varios desarrolladores oficiales del proyecto en el
propio equipo de trabajo.
% </+>

Si tenemos un buen conocimiento de un producto podemos vender consultoría,
adaptación, integración, etc\ldots

Intentar que alguien use mucho el software y luego me paguen por hacer
modificaciones o añadir servicios.

Algunos casos, pueden ser:
\begin{itemize}
  \item Levanta, dan consultoría y soporte GNU/Linux y software libre en EEUU.
  \item Alcove, daba consultoría y consultoría estratégica para software libre
  en Europa.
\end{itemize}

% <+>
Generalmente, para implantar este modelo, se suele hacer una inversión
inicial muy fuerte y se espera que el dinero vaya llegando con el
tiempo. Por ser una práctica con cierto riesgo, se suelen realizar
estudios de mercado previos.
% </+>

\subsection{Mejor conocimiento con limitaciones}

El tener el mejor conocimiento tiene la limitación de que el resto también
puedan conseguirlo. Aunque existe otra forma que se llama tener
mejor conocimiento con limitaciones.

Este modelo intenta limitar el problema que teníamos anteriormente de que la
competencia también sea experta en el mismo producto que tú. Para ello se
utilizan licencias, en las que tienen una parte privativa y una parte libre. De
manera que la parte privativa esta más enfocada para dar servicio a empresas y
éstas tienen que pagar y una parte libre más enfocada para todo el mundo.
Normalmente la parte privada tendrá componentes o servicios que no tendrá la
libre, y así potencias que paguen la licencia.

El problema que tiene esto es que las comunidades, pueden crear las partes
que falten en la parte privativa y eso te hará perder marca y ser el productor
principal. Como es el caso de OpenOffice y LibreOffice.

% <+> 
Otro contexto en el que se utiliza este modelo es cuando una empresa o
individuo es poseedor de cierta patente, lo que imposibilita
legalmente que determinada parte pueda ser reimplementada por terceras
personas.
% </+>

Los proveedores de CRM hacen cosas de este estilo.


\subsection{Fuente de un programa}

Ser la fuente de un programa. Es la idea de tener una marca y ser el punto de
referencia sobre un producto. Es más terminos de imagen, como Sun con OpenOffice.

Otro ejemplo claro es IBM que financia Eclipse y esto da una sensación de que
IBM hace cosas serias. Presumiblemente, este sería el mismo caso de Sun
Microsystems con OpenOffice, es decir, la búsqueda de una buena imagen de
marca, y más teniendo en cuenta que no está del todo claro que Sun llegase
nunca a tener ingresos directos en concepto de OpenOffice. Además, se debe
tener en cuenta que apadrinar un producto como Eclipse en el caso de IBM, a
veces puede resultar menos costoso que el hecho de obtener publicidad mediante
vías más tradicionales.

Finalmente la financiación se termina rentabilizando en términos de imagen de
marca.

Algunos ejemplos más de esto son:
\begin{itemize}
 \item Abiword, intentó crear una suite ofimática, pero cuando OpenOffice se liberó,
pues no consiguieron seguir.
\item Evolution, RedCarpet, creada por Ximian, las regalaban en las
aplicaciones, finalmente se vendieron a Novel, para recuperar el dinero
invertido para crear Evolution. Mucha gente les conocía y estaban en una buena
posición dentro del mercado.
\item Zope, tenía un producto privativo, cuando fueron a pedir financiación, la
empresa les dijo que sólo les financiaba si lo hacían libre, de esta manera se
ha creado una gran comunidad alrededor de Zope.
\item Por último, el caso de Asterisk, que se espera que sea el centro para la
gestión de centralitas para su uso en VoIP.
\end{itemize}



\subsection{Fuente de un programa con limitaciones}

Otra forma de poner limitaciones a la competencia es que primero puedo hacer el
producto privativo y cuando me interese lo libero, de esta manera la competencia
irá retrasada con respecto a mí.

Un caso claro es lo que está haciendo Google con la última versión de Android,
primero se la facilita a los fabricantes de móviles para que puedan adaptarlas a
su manera, y cuando éstos ya están disponibles en el mercado, libera la versión
para el resto del mundo.

Aunque hay que decir que esto no siempre funciona.

\subsection{Licencias especiales}
Es una variante del apartado anterior, lo único es que lo hago a la vez.
Pero la versión privativa me permite crear trabajos
derivados de ella y puedo ponerle la licencia que yo quiera. En cambio, la
versión libre siempre llevará una licencia GPL y si alguien realiza una versión
derivada, tendrá que añadir esta licencia.

Según el modelo de negocio dependerá mucho de las licencias que uses.

\subsection{Venta de marca}
Lo único que quiero es vender marca, del estilo de Red-Hat. Vende servicios
alrededor de su marca, sólo por tener la marca la gente ya va a pagar porque
tienes una imagen detrás que respalda el trabajo que realizas.

\section{Desarrollos sin financiación directa}
Es posible que existan proyectos en los que haya desarrolladores que sus
propias empresas les dejan trabajar para otros proyectos de software libre, bien
en su tiempo libre o bien ciertas horas al día.

Aunque no están financiados sí reciben alguna contribución, como parches por
alguna cierta funcionalidad, donación de máquinas como infraestructura,
donaciones propias, etc\ldots


\section{Desarrollos para uso internos}
Este tipo de financiación está basada en que yo necesito algo y lo desarrollo como
software libre, de esta manera al publicarse, el código estará más limpio y en un
futuro puedo encontrar colaboración.

% <+>
Un ejemplo de este modelo es el que utilizó Cisco en un sistema de
gestión de impresoras. Llegaron a la conclusión de que no perderían
nada con su liberación y que era una manera de mejorarlo. Otro efecto
colateral era sobre el ánimo de los desarrolladores, ya que estaban
más motivados por trabajar en un proyecto con visibilidad externa.
% </+>


\section{Clasificación de Hecker}

A lo largo de este capítulo se ha estudiado una posible clasificación para los
diversos modelos de negocio entorno al \textit{software} libre. Pero por
supuesto, existen múltiples clasificaciones, entre las cuales destaca la
realizada por Hecker y que fue adoptada por la \textit{Open Source Initiative}.
Quizás la peculiaridad de esta clasificación reside en que para determinados
modelos, el \textit{software} no ocupa un lugar de verdadera relevancia sino que
más bien supone un acompañante que ofrece valor añadido al producto
final\footnote{Tabla extraída del libro ``\textit{Handbook of Research on Open
Source Software: Technological, Economic, and Social Perspectives}'', de Kirk
St. Amant y Brian Still}. Pese a todo, se pueden encontrar correspondencias
directas entre algunos de los modelos de la clasificación de Hecker y
algunos otros de los estudiados con detenimiento a lo largo del capítulo.

\begin{figure}[htb]
  \centerline{\resizebox{\textwidth}{!}{\includegraphics{img/hecker}}}
  \label{IMG:Hecker}
\end{figure}

\section{Otros modos de financiación}

Existen otros modos de financiación cuya cabida dentro de alguna de las
categorías estudiadas en las clasificaciones anteriores resulta complicada.
Algunas de ellas pueden ser:

\begin{itemize}
  \item Mediante sitios en los que se ponen en contacto desarrolladores y
clientes, de tal modo que un desarrollador indica su perfil y habilidades,
mientras que los clientes publican sus necesidades, de tal modo que puedan
llegar a un acuerdo contractual para el producto. Un claro ejemplo de este
modelo se ejemplifica a través de SourceXchange.
  \item Mediante venta de bonos que cotizasen al estilo de una bolsa o mercado
de bonos. El funcionamiento simplificado de una de sus variantes más extendidas
sería el siguiente: un desarrollador tiene una idea para un nuevo programa o
mejora de otro ya existente, ante lo cual, indica sus requisitos,
especificación, presupuesto, etc. Se generan unos bonos y en caso de que el
presupuesto indicado haya sido cubierto (el desarrollador ha vendido los bonos
suficientes como para ello), el desarrollador comienza a trabajar en ese
software, teniendo siempre en cuenta que si el proyecto no llega a ejecutarse,
la donación en forma de esos bonos es devuelta a sus depositarios. De esta
manera comenzó el trabajo en la red social distribuida Diaspora, mediante este
modelo que ha dado en llamarse financiación de las masas (Cloud Sourcing) y que
es un ejemplo más de las nuevas formas de colaboración que permite la
tecnología.
  \item Mediante cooperativas de desarrolladores, de modo que los
desarrolladores de software libre, en lugar de trabajar individualmente, se
reúnen a través de algún tipo de asociación similar a una cooperativa y con un
funcionamiento parecido al de una empresa, salvando el carácter ético con el
mundo del software libre. Un ejemplo de este tipo de organizaciones es Free
Developers.
  \item Mediante un sistema de donaciones (similar a la venta de bonos) en el
que los usuarios interesados en que un proyecto continúe lanzando nuevas
versiones, pueden realizar un pago/donación a su autor de manera voluntaria a
través de su página web y sirviendo como método de financiación para dicho
desarrollador.
\end{itemize}

\section{Modelos Mixtos}

A modo de resumen, cabe resaltar que en la práctica, las empresas llevan a cabo
estrategias basadas en la combinación de varios de los modelos de negocio aquí
vistos, dándose origen por lo tanto a modelos mixtos. Es más, tanto es así, que
en raras excepciones esas empresas se dedican únicamente al software libre
(quizás el más claro de los ejemplos sea el caso de Google, que si bien
capitanea la plataforma Android, no libera el código fuente y algoritmos de su
buscador y demás servicios en línea). En cualquier caso, es muy destacable
analizar cómo cada vez más empresas consideradas como ``tradicionales'' en el
mundo de la computación, cada vez con más frecuencia prueban líneas de negocio
basadas en el software libre o lanzan proyectos de este tipo como producto
estratégico en un determinado mercado, así como el hecho mismo de intentar
fortalecer la imagen de marca.

\section{Ley de Conservación de los Beneficios Atractivos}

Prácticamente a modo de anexo de este capítulo, presentamos aquí la llamada Ley
de Conservación de Beneficios Atractivos, enunciada por Clayton Christensen y
que dice así:

\textit{``Cuando desaparecen los beneficios atractivos en una etapa de la cadena
de valor porque un producto se hace modular y ''commodity``, la oportunidad para
conseguir beneficios atractivos con productos privativos normalmente aparecerá
en una etapa adyacente.''}

Pero antes de nada, se debe tener en cuenta que un producto pasa por diferentes
fases de la cadena de producción, cada una de las cuales además de sumar un
determinado valor al producto final, también supone un coste adicional al
mismo. Así pues, lo que enuncia esta ley es que si se consigue que uno de los
eslabones de la cadena de valor desaparezca, o lo que es lo mismo, tenga
coste nulo, no por ello el precio final del producto se va a ver reducido sino
que se va a mantener dado que dicho precio de mercado viene establecido por la
ley de la oferta y de la demanda, de tal modo que el beneficio en esa etapa de
la cadena que se ha conseguido hacer modular y ``commodity'' pasará a las etapas
adyacentes.

Así, el software libre es considerado como un factor de ``comoditización'',
permitiendo la transferencia de oportunidades de beneficio a su alrededor, lo
que lo convierte por tanto en un elemento clave al que las empresas deben
prestar una especial atención de cara a sus estrategias de negocio.

\chapter{Desarrollo de software libre. Organizaci\'on de las comunidades}
\label{CHAP5:Development}
\section{Introducci\'on}
Las reglas de gobierno de las comunidades son muy diversas. Las normas en el software libre tienen que ser muy inclusivas, es decir tienen que estar de acuerdo, para que las personas sigan en la comunidad contribuyendo.

El factor primero es el tama\~no, ya que si es peque\~no normalmente no tiene organizaci\'on, pero si es muy grande seguramente tiene una organizaci\'on. Cuando una persona quiere contribuir tiene que ser consciente de la organizaci\'on. Las normas pueden aparecer o no escritas en la documentaci\'on de la organizaci\'on. Si uno es desarrollador debe de cumplir las normas del proyecto. Todo esto se aplica a la econom\'ia de atenci\'on.

Vamos a estudiar las organizaciones desde el punto de vista de la organizaci\'on del proyecto. Seguidamente se ver\'a como son las condiciones de contribuci\'on y como se fomentan las contribuciones externas.

A continuaci\'on vamos a estudiar algunos proyectos.

\section{Proyecto Apache (ASF)}
\section{OpenOffice}
\section{Mozilla}
\chapter{An\'alisis de proyectos}
\label{CHAP6:Analysis}
\section{Introducci\'on}
Los proyectos de software libre son p\'ublicos por naturaleza y podemos tener una serie de datos que podemos encontrar y comparar con otros proyectos.

Como todo an\'alisis emp\'irico tenemos que seguir una metodolog\'ia:
\begin{enumerate}
\item
El proyecto que queremos estudiar.
\item
Identificar la fuente de datos. Seguir el sistema de gesti\'on de incidencias, recuperar los datos, y una vez que hemos realizado la gesti\'on de incidencias y la recuperaci\'on de datos tenemos que realizar la miner\'ia de datos, es decir, realizar una t\'ecnica descriptiva de un proyecto.
\item
Realizar el an\'alisis. Extraecci\'on la informaci\'on que no es evidente, aport\'andonos m\'as informaci\'on. Por ejemplo, en que meses del a\~no se realiza m\'as trabajos en el proyecto.
\item
Realizar un informe explicativo. Una vez realizado nuestro an\'alisis tenemos que componer un informe que explica toda la informaci\'on del proyecto.
\end{enumerate}

\section{Herramientas para el an\'alisis}
Para realizar un an\'alisis de proyecto necesitamos un conjunto de herramientas autom\'aticas para evitar errores, para reducir tiempos, y para la replicabilidad. La replicabilidad nos permite construir un proyecto a partir de otro y no empezar de cero ahorr\'andonos un valioso tiempo. 
Es deseable que las herramientas que vamos a utilizar sean software libre porque es m\'as f\'acil de integrarlas, extenderlas y aplicarla a nuevas funcionalidades. 
Para realizar cualquier tipo de an\'alisis estad\'istico tenemos una librer\'ia de R. R es un software libre que sirve para el análisis estad\'istico y gr\'afico en un entorno de programaci\'on. Adem\'as nos permite cargar diferentes bibliotecas o paquetes con finalidades espec\'ificas de c\'alculo o gr\'afico. 
Por otro lado necesitamos soporte de administraci\'on de herramientas estad\'isticas, esto se traduce a tener un ordenador potente o un servidor potente (gran capacidad de disco duro, ram, cpu potente, etc).

Qu\'e podemos usar o aplicar para realizar un an\'alisis:
\begin{enumerate}
\item
Vamos a utilizar una base de datos, MySQL/SQLite.
\item
Una herramienta que extrae la informaci\'on de c\'odigo fuente de los registros y la almacena en una base de datos, CVSAnalY.
\item
Y un analizador estad\'istico, GNU R.
\end{enumerate}

Para la estaci\'on de datos vamos a extraer informaci\'on de repositorios p\'ublicos como son CVS, SVN y GIT. Nos proporciona un mecanismo autom\'atico que apuntando a una url nos trae la informaci\'on, la parsea y la guarda en una base de datos o en un fichero para poder trabajar.

Podemos extraer datos para las acciones de desarrollo, lo vamos a tener en una tabla de la base de datos con nombre "scmlog".
Tenemos datos registrados de archivos que est\'a en la tabla "file".
Otra tabla para los datos de las personas involucradas en el proyecto "people".

\section{Algunos an\'alisis}



\end{document}
