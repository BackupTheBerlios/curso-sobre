\chapter{Introducción al \textit{Software Libre}}
\section{Introducción y Motivación al \textit{Software Libre}}

TODO: by jfernandez

\section{Historia del \textit{Software Libre}}

TODO: by jfernandez

\section{La Definición de \textit{Open Source}}

\subsection{Motivación}

Existe un motivo fundamental para definir aquello de lo que estamos hablando:
evitar confusiones. Por ello, en esta sección estableceremos a qué nos estamos
refiriendo al hacer mención de aspectos como \textit{Free Software},
\textit{Open Source} o \textit{Software Libre}.

Para ello, debemos recurrir en primera instancia a Richard Stallman y la
\textit{Free
Software Definition}\footnote{http://www.gnu.org/philosophy/free-sw.html}, cuya
lectura detenida queda totalmente recomendada. En esta definición, se establecen
las cuatro libertades que debe ofrecer un \textit{software} para que pueda ser
considerado como libre, y que se detallan a continuación:

\begin{enumerate}
  \setcounter{enumi}{-1}
  \item La libertad de ejecutar el programa para cualquier propósito. De lo
cual se desprende que no se debería prohibir el uso del programa, por ejemplo,
con fines militares, si bien es cierto que esto no ampara los posibles usos para
fines ilegales.
  \item La libertad para estudiar cómo funciona el programa y modificarlo como
se desee. De esto se desprende lo que se podría considerar como un corolario:
es necesario tener acceso al código fuente.
  \item La libertad para redistribuir copias ya que con ello, puedes ayudar a
tu vecino. La manera de redactar esta libertad en concreto, deja entrever la
filosofía de vida y pensamiento de Richard Stallman.
  \item La libertad para redistribuir copias del \textit{software} una vez
modificado. Se puede entender como una combinación de las dos libertades
anteriores.
\end{enumerate}

Como análisis de estas libertades, se puede afirmar que Richard Stallman en el
momento de su definición, estaba poniendo el énfasis en la idea de la libertad
(en el sentido de la libertad de expresión). De hecho, en ningún momento se
recogen menciones al ámbito comercial y/o económico, ya que para tener libertad
de expresión no es necesario pagar (obsérvese la desambiguación del término
inglés \textit{``free''}, indicándose que debe entenderse en el sentido de
\textit{``free speech''} y nunca en el sentido de \textit{``free beer''}, por lo
que igualmente se hace énfasis de una manera indirecta en que por el hecho de
ser libre, el \textit{software} no tiene por qué ser gratuito).

En cualquier caso, la \textit{Free Software Definition} es tan sólo una
definición que nos ayuda a identificar y concretar a qué nos referimos al
emplear el término \textit{Free Software}. Pero a la hora de materializar este
concepto en la práctica, ¿cómo ha de hacerse? Debido a la propiedad intelectual
y sus mecanismos, si no se da permiso explícito, esa obra pertenece a su creador
y nadie más puede hacer uso de ella.

De aquí nace la necesidad de contar con el concepto de licencia, mecanismo
empleado para indicar al usuario qué puede hacer y qué no con ese
\textit{software}, y que se estudiará en detalle en el capítulo
\ref{CHAP2:Licenses} de este documento. De este modo, será necesario
leer la licencia de ese \textit{software} y en caso de cumplirse con las cuatro
libertades indicadas anteriormente, nos encontraremos ante un \textit{software}
libre. De lo contrario (bastará con que no se cumpla tan sólo una de las cuatro
libertades) estaremos ante un \textit{software} privativo o \textit{software}
propietario (estos dos últimos términos se emplean como sinónimos, pese a que lo
correcto es el término privativo, ya que por el hecho de hacer que un
\textit{software} sea libre, no se pierde la autoría desde el punto de vista
legal, se continúa siendo el propietario de cara a la ley).

En cualquier caso, resulta indudable que la lectura y análisis de una licencia
\textit{software} es un trabajo tedioso. Por ello, es preferible contar con un
listado de licencias asimiladas y reconocidas como de \textit{software} libre.
Además, este listado de licencias deberá ser emitido por una entidad de
reconocido prestigio en la materia.

Retomando la definición ofrecida por la \textit{Free Software Foundation}, debe
notarse que se trata de un lenguaje abstracto y de altísimo nivel, mientras que
una licencia emplea un lenguaje muy concreto de lo que se puede y lo que no se
puede hacer. Este fue el motivo por el que nacieron las \textit{Debian Free
Software Guidelines}\footnote{http://www.debian.org/social_contract#guidelines}
(lectura totalmente recomendada): dado que la definición original es de muy alto
nivel, se deben establecer reglas más detalladas, de tal modo que se pase de
hablar de aspectos como la libertad de expresión, a cosas mucho más concretas.

Este trabajo llevado a cabo por \textit{Debian} tuvo gran éxito y buena acogida,
y por ello fue tomado como base para otras definiciones, como es el caso de
la \textit{Open Source Definition}. Pero, ¿por qué es necesaria una nueva
definición para una misma cosa? La razón de esto se encuentra en que pese a
todas las matizaciones e indicaciones, \textit{Free Software} puede referirse en
inglés tanto a \textit{``libre''} como a \textit{``gratis''}. Por ello comenzó
también a emplearse el término \textit{Open Source}, dándose origen a la
\textit{Open Source Definition}, que vamos a estudiar en detalle a continuación.

