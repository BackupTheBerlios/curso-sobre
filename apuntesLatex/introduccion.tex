\chapter{Introducción al \textit{Software Libre}}
\section{Introducción y Motivación al \textit{Software Libre}}

Navegadores \textit{web}, gestores de correo electrónico, juegos, herramientas
para edición de vídeo, reproductores multimedia, suites ofimáticas, aplicaciones
de mensajería instantánea, analizadores de tráfico, servidores \textit{HTTP},
...

Actualmente el \textit{software libre} nos ofrece todo tipo de alternativas
tanto en el segmento de servidores como en aplicaciones de escritorio que
estamos acostumbrados a usar a diario. Pero, ¿qué es el \textit{software libre}?
¿Cuándo nació? ¿Cómo ha evolucionado? A estas y a algunas otras preguntas
intentaremos responder a lo largo de este primer capítulo de la asignatura.

\subsection{El \textit{Software Libre} como Alternativa}

El modelo tradicional de producción ha conllevado a ciertas anomalías, en el
fondo, todas provenientes de la idea de que tan sólo el productor puede mejorar
su producto.

Por otro lado, el productor impone ciertas condiciones a través de
la licencia del \textit{software}, licencia que se deberá aceptar con tal de
poder hacer uso del programa, hasta el punto de que en ocasiones, en el momento
de realizar la compra estamos aceptando los términos de la licencia sin ser
conscientes de ello.

Otra idea heredada a lo largo de los años es que tan sólo se podrá hacer
negocio (ingresos y ventas) si se tiene el control absoluto sobre la copia y
redistribución. Se trata al mismo tiempo de un modelo de negocio basado en la
exclusividad, de manera que un único productor ofrece ese programa,
produciéndose por tanto monopolios (sabido es el ejemplo de \textit{Microsoft
Office}, empleado mayoritariamente de una manera muy probable por tradición y
por desconocimiento de alternativas). De este modo, se produce un
monopolio no ya de producto, sino que llega a asociarse ese determinado
producto con la empresa que lo produce. Además, el productor será el único con
la potestad para resolver problemas del \textit{software} y mejorarlo, con lo
que se frena también el proceso de mejora e innovación.

Así pues, el \textit{software} libre supone, en cierto modo, un punto de ruptura
con respecto a esta visión tradicional dadas las características que presenta y
que estudiaremos a continuación.

\subsection{¿Qué es el \textit{Software Libre}?}

En el espectro del mundo del \textit{software}, se puede afirmar que
\textit{software} libre y \textit{software} privativo son antagonistas, extremos
contrarios el uno respecto al otro.

Al contrario de lo visto en la sección anterior para el caso del modelo
tradicional, el \textit{software} libre presta una atención muy especial a la
redistribución y modificación del producto. Es más, el usuario que recibe el
\textit{software} libre, podrá:

\begin{itemize}
  \item Usarlo cómo y dónde quiera.
  \item Redistribuirlo a quien quiera y de la manera que quiera.
  \item Modificarlo (para mejorarlo, adaptarlo, etc.).
  \item Redistribuir esas modificaciones.
\end{itemize}

Estos cuatro puntos se corresponden realmente con las cuatro libertades del
\textit{software} libre y que se estudiarán en detalle en la sección
\ref{SUBSEC:DefinicionMotivacion} de este mismo capítulo.

Tal y como se puede entender de los cuatro puntos anteriores, será necesario
contar con el código fuente.

Y por último, si hay un falso mito en torno al mundo del \textit{software}
libre, sobre lo que es y lo que no es, ante todo, debe quedar claro lo
siguiente: \textit{software} libre no es igual a \textit{software} gratis. Esta
confusión se ha producido a lo largo de los años por el hecho de que
habitualmente, los usuarios pueden descargarse el software libre de una manera
gratuita. Es más, en inglés el término ``free'' puede hacer referencia tanto a
``libre'' como a ``gratis'', por lo que se genera ambigüedad que si bien en
la lengua española no existe, sí se cuenta con esa misma confusión por razones
distintas. Este mismo hecho, motivó la aparición de un nuevo término para
referirse a lo mismo, tal y como se estudiará más adelante: \textit{Open
Source}.

Debido a esas libertades con que contará el \textit{software} libre frente a
otros modelos, se tienen una serie de consecuencias como que por ejemplo, el
modelo de costes será totalmente distinto al del \textit{software} privativo.
También podrá inspeccionarse, estudiarse, modificarse, etc., cosa impensable
con el \textit{software} privativo ya que normalmente, es distribuido en forma
binaria a través de sus ejecutables. También surgen nuevos canales y métodos de
distribución, al mismo tiempo que los modelos de desarrollo son a la vez que
revolucionarios, hasta cierto punto sorprendetes; es más, a menudo se incumplen
las normas básicas que tradicionalmente se han tenido en cuenta como prácticas
ideales en ingeniería del \textit{software}, y no por ello, los resultados
finales son de mala calidad. Por último, surge también una fuerte competencia
en el campo del mantenimiento y soporte del \textit{software} libre. Es decir,
que de esta manera, se combinan dos poderosos mecanismos: la competencia y la
cooperación (a menudo, incluso de una manera voluntaria).

Así pues, en este nuevo escenario que se presenta ante el mundo y filosofía del
\textit{sofware} libre, aparecen un total de hasta cuatro actores: los usuarios
finales del \textit{software} desarrollado, que podrán ser tanto particulares
como empresas; el desarrollador o productor del \textit{software}; el integrador
de sistemas con el fin último de desarrollar soluciones prácticas y eficaces; y
por el último, el actor encargado de ofrecer mantenimiento, servicio y soporte a
ese \textit{software}.

Al mismo tiempo, alrededor del \textit{software} libre aparece todo un listado
de términos que pueden causar sobre lo que verdaderamente es o no es. Del ya
comentado caso del término ``\textit{Open Source Software}'' que se estudiará en
mayor profundidad a lo largo del capítulo, podemos decir que en general, y salvo
excepciones muy puntuales, el \textit{software open source} es también
\textit{software} libre. Pero a este término se suma también la idea del
dominio público, en el cual se debe tener claro que el autor realmente está
cediendo sus derechos a la sociedad. O el término \textit{copyleft}, que hace
referencia a que ese \textit{software}, tras una serie de redistribuciones,
mantendrá las libertades para quien lo reciba.

\subsection{Conclusiones}

¿Qué conclusiones existen en la actualidad respecto al mundo del
\textit{software} libre? Lo cierto es que realmente todavía no se cuenta con la
suficiente experiencia como para aventurarse a lanzar un veredicto sobre las
tendencias futuras en el \textit{software} libre. Sin embargo, sí existen
perspectivas interesantes de cada al futuro, como por ejemplo la buena muestra
de casos en los que se ha demostrado la viabilidad económica y técnica del
\textit{software} libre. El modelo favorece a los más competitivos, ya que
permite estar en la cresta de la ola tecnológica a todos aquellos con verdaderas
ganas de innovar y para lo cual es necesario tener un buen (y actualizado)
conocimiento del área, facilitando el trabajo y ofreciendo posibilidades incluso
a los más pequeños dentro del mercado.

No obstante, todavía quedan problemas por resolver y de los cuales, quizás
puedan surgir nuevas oportunidades, así como técnicas con las que experimentar
en muy diversas áreas, desde los modelos de desarrollo hasta los modelos de
negocio. En cualquier caso, lo que sí parece claro es que se trata de un momento
en el que toda la industria puede estar cambiando de paradigma y concepción
acerca del sector, como de hecho bien demuestra la cantidad de grandes
multinacionales que cada vez con mayor frecuencia e implicación, se acercan al
\textit{software} libre.

\section{Historia del \textit{Software Libre}}

TODO: by jfernandez

\subsection{Primeros Años}

Durante los primeros años de la computación (décadas de 1950 y 1960), la
situación del \textit{software} dentro del mundo de la computación era muy
similar a la filosofía que hoy conocemos bajo el término ``\textit{software
libre}''. En aquellos años, el \textit{software} (incluyendo su código fuente)
estaba incluído de manera inherente a la adquisición del \textit{hardware},
hasta el punto que el \textit{software} no era ni visto ni considerado como una
pieza independiente del \textit{hardware}, es decir, que el \textit{software}
era una especie de acompañante. Debido a esto, los profesionales de la
computación podían mejorar (y así lo hacían) ese \textit{software},
compartiéndolo con otros grupos. Es más, siempre que se pagase el contrato, se
tenía acceso al catálogo de \textit{software} del fabricante. Sin embargo, esta
situación cambió radicalmente el 30 de junio de 1969, cuando \textit{IBM}
(principal fabricante y con mucha diferencia sobre sus competidores) anunció que
a partir de 1970, dejaría de incluir el \textit{software} con sus computadores y
comenzó a venderlo de manera independiente. Fue entonces cuando se empezó a ver
el \textit{software} como un producto con valor intrínseco, y con ello, se
limitaron y restringieron sobremanera las posibilidades que sus usuarios tenían
de estudiar el código, mejorarlo, compartirlo, etc.

Con esta situación, a mitad de la década de 1970 ya era completamente habitual
encontrarse con \textit{software} privativo, lo que supuso un fuerte cambio
cultura entre los profesionales del sector. Pero pese a que la tendencia era
hacia la exploración del nuevo modelo ofrecido por el \textit{software}
privativo, existieron durante estos años iniciativas importantes que encajan a
la perfección con lo que hoy conocemos como \textit{software} libre mediante las
cuales se perseguía una herramienta determinada, amparada ya fuera por motivos
éticos (costumbre en la comunidad matemática) o prácticos (difusión científica).
Algunos ejemplos a destacar son \textit{SPICE} para la simulación de circuitos
integrados, \textit{TeX} como sistema de tipografía electrónica y el complejo
caso de \textit{Unix}, sistema operativo portable originalmente creado con Ken
Thompson y Dennis Ritchie en los \textit{Bell Labs} de \textit{AT&T}.

Pero hasta el momento todo fueran iniciativas prácticamente individuales. No
fue hasta comienzo de la década de 1980 cuando comenzaron a aparecer proyectos
organizados y que de una manera consciente, buscaban sistemas compuestos
únicamente por \textit{software} libre. Fue precisamente durante estos años
cuando se asentaron las bases y fundamentos filosóficos y legales (a través de
la licencia \textit{GPL}) del movimiento del \textit{software} libre. Richard
Stallman al frente, como padre ideológico y cabeza visible del movimiento, dejó
en 1984 su puesto de trabajo en el \textit{MIT} para comenzar a trabajar en el
proyecto \textit{GNU}, con la idea de crear un sistema \textit{software} de
propósito general y completamente libre. De hecho, aportó alguna infraestructura
básica para continuar trabajando en el sistema, como fueron el editor de textos
\textit{Emacs} y el importante compilador de \textit{C} llamado \textit{GCC}.

Además, Richard Stallman también se encargó de escribir la licencia \textit{GPL}
con la que se garantizaba que cualquier usuario, tras un número indefinido de
redistribuciones, siguiese contando con las mismas libertades para modificar el
\textit{software}, redistribuirlo, etc. Y del mismo modo, fundó la \textit{Free
Software Foundation} con tal de conseguir fondos destinados al desarrollo y
protección del \textit{software} libre, asentándose las bases del pensamiento y
la ética en el \textit{The GNU Manifesto}.

Por otro lado, alrededor de \textit{Unix} se creó una comunidad de
desarrolladores que pronto tuvo como centro al \textit{CSRG} de la
\textit{Universidad de California} en Berkeley, hasta el punto de llegar a
convertirse en una de las dos fuentes principales de \textit{Unix}, junto con la
oficial, \textit{AT&T}. De hecho, tal fue el éxito del \textit{software}
desarrollado en el \textit{CSRG} que tras liberaciones de su código del núcleo y
todas las utilidades de un sistema \textit{Unix} completo, se daría origen
primero al sistema \textit{386BSD} (Bill Jolizt escribío el código fuente del
núcleo para que funcionase sobre \textit{i386}), y a partir de ahí nacería toda
la familia \textit{*BSD} (\textit{NetBSD}, \textit{FreeBSD}, \textit{OpenBSD}),
siempre distribuida bajo licencias tipo \textit{BSD} y que en muchas ocasiones
fue y sigue siendo utilizado por software privativo: \textit{SunOS},
\textit{Ultrix}, etc.

Un hito importante en la historia temprana del \textit{software} libre lo supuso
\textit{Internet}, desde su nacimiento a comienzos de la década de 1970 y con el
que estableció una estrecha relación. Fue importante la distribución de una
implementación libre de la pila \textit{TCP/IP} llevada a cabo por \textit{BSD
Unix} y fue tomada como la de referencia. La red pronto se convirtió en una
herramienta fundamental, al permitir la cooperación y compartir información y
\textit{software} a través de las mismas y con independencia de la distancia
existente entre los interesados. A todo esto pronto se añadieron nuevas
herramientos que ayudaron aún más en la metodología de trabajo, como
\textit{News}, \textit{FTP}, correo electrónico, etc. Y con todo ello,
comenzaron a formarse comunidades alrededor de proyectos de \textit{software}
libre tal y como hoy las conocemos.


\section{La Definición de \textit{Open Source}}

\subsection{Motivación}
\label{SUBSEC:DefinicionMotivacion}

Existe un motivo fundamental para definir aquello de lo que estamos hablando:
evitar confusiones. Por ello, en esta sección estableceremos a qué nos estamos
refiriendo al hacer mención de aspectos como \textit{Free Software},
\textit{Open Source} o \textit{Software Libre}.

Para ello, debemos recurrir en primera instancia a Richard Stallman y la
\textit{Free
Software Definition}\footnote{http://www.gnu.org/philosophy/free-sw.html}, cuya
lectura detenida queda totalmente recomendada. En esta definición, se establecen
las cuatro libertades que debe ofrecer un \textit{software} para que pueda ser
considerado como libre, y que se detallan a continuación:

\begin{enumerate}
  \setcounter{enumi}{-1}
  \item La libertad de ejecutar el programa para cualquier propósito. De lo
cual se desprende que no se debería prohibir el uso del programa, por ejemplo,
con fines militares, si bien es cierto que esto no ampara los posibles usos para
fines ilegales.
  \item La libertad para estudiar cómo funciona el programa y modificarlo como
se desee. De esto se desprende lo que se podría considerar como un corolario:
es necesario tener acceso al código fuente.
  \item La libertad para redistribuir copias ya que con ello, puedes ayudar a
tu vecino. La manera de redactar esta libertad en concreto, deja entrever la
filosofía de vida y pensamiento de Richard Stallman.
  \item La libertad para redistribuir copias del \textit{software} una vez
modificado. Se puede entender como una combinación de las dos libertades
anteriores.
\end{enumerate}

Como análisis de estas libertades, se puede afirmar que Richard Stallman en el
momento de su definición, estaba poniendo el énfasis en la idea de la libertad
(en el sentido de la libertad de expresión). De hecho, en ningún momento se
recogen menciones al ámbito comercial y/o económico, ya que para tener libertad
de expresión no es necesario pagar (obsérvese la desambiguación del término
inglés \textit{``free''}, indicándose que debe entenderse en el sentido de
\textit{``free speech''} y nunca en el sentido de \textit{``free beer''}, por lo
que igualmente se hace énfasis de una manera indirecta en que por el hecho de
ser libre, el \textit{software} no tiene por qué ser gratuito).

En cualquier caso, la \textit{Free Software Definition} es tan sólo una
definición que nos ayuda a identificar y concretar a qué nos referimos al
emplear el término \textit{Free Software}. Pero a la hora de materializar este
concepto en la práctica, ¿cómo ha de hacerse? Debido a la propiedad intelectual
y sus mecanismos, si no se da permiso explícito, esa obra pertenece a su creador
y nadie más puede hacer uso de ella.

De aquí nace la necesidad de contar con el concepto de licencia, mecanismo
empleado para indicar al usuario qué puede hacer y qué no con ese
\textit{software}, y que se estudiará en detalle en el capítulo
\ref{CHAP2:Licenses} de este documento. De este modo, será necesario
leer la licencia de ese \textit{software} y en caso de cumplirse con las cuatro
libertades indicadas anteriormente, nos encontraremos ante un \textit{software}
libre. De lo contrario (bastará con que no se cumpla tan sólo una de las cuatro
libertades) estaremos ante un \textit{software} privativo o \textit{software}
propietario (estos dos últimos términos se emplean como sinónimos, pese a que lo
correcto es el término privativo, ya que por el hecho de hacer que un
\textit{software} sea libre, no se pierde la autoría desde el punto de vista
legal, se continúa siendo el propietario de cara a la ley).

En cualquier caso, resulta indudable que la lectura y análisis de una licencia
\textit{software} es un trabajo tedioso. Por ello, es preferible contar con un
listado de licencias asimiladas y reconocidas como de \textit{software} libre.
Además, este listado de licencias deberá ser emitido por una entidad de
reconocido prestigio en la materia.

Retomando la definición ofrecida por la \textit{Free Software Foundation}, debe
notarse que se trata de un lenguaje abstracto y de altísimo nivel, mientras que
una licencia emplea un lenguaje muy concreto de lo que se puede y lo que no se
puede hacer. Este fue el motivo por el que nacieron las \textit{Debian Free
Software Guidelines}\footnote{http://www.debian.org/social_contract#guidelines}
(lectura totalmente recomendada): dado que la definición original es de muy alto
nivel, se deben establecer reglas más detalladas, de tal modo que se pase de
hablar de aspectos como la libertad de expresión, a cosas mucho más concretas.

Este trabajo llevado a cabo por \textit{Debian} tuvo gran éxito y buena acogida,
y por ello fue tomado como base para otras definiciones, como es el caso de
la \textit{Open Source Definition}. Pero, ¿por qué es necesaria una nueva
definición para una misma cosa? La razón de esto se encuentra en que pese a
todas las matizaciones e indicaciones, \textit{Free Software} puede referirse en
inglés tanto a \textit{``libre''} como a \textit{``gratis''}. Por ello comenzó
también a emplearse el término \textit{Open Source}, dándose origen a la
\textit{Open Source Definition}, que vamos a estudiar en detalle a continuación.

\subsection{\textit{The Open Source Definition
(Annotated)}\footnote{http://www.opensource.org/docs/definition.php}}

Tomando como base las \textit{Debian Free Software Guidelines}, y con tal de
desambiguar el término \textit{``free''} en inglés (contando con la polisemia
\textit{``gratis''} y \textit{``libre''}), nació la \textit{Open Source
Initiative} (en un primer momento, intentando incluso la creación de
\textit{Open source} como marca comercial, cosa que no se consiguió debido a que
se trataba de un término demasiado genérico) con su \textit{Open Source
Definition} que será analizada durante esta sección, y que comienza de la
siguiente manera:\newline

{\bf Introduction

Open source doesn't just mean access to the source code. The distribution terms
of open-source software must comply with the following criteria:}\newline

Aquí se aclara que para que un \textit{software} se pueda considerar como
\textit{Open source}, no se debe cumplir únicamente con que se tenga acceso a su
código fuente (hecho que por otro lado, también genera ciertas confusiones en la
actualidad pese a la clara matización que se ofrece).\newline

{\bf 1. Free Redistribution

The license shall not restrict any party from selling or giving away the
software as a component of an aggregate software distribution containing
programs from several different sources. The license shall not require a royalty
or other fee for such sale.}

\textit{Rationale: By constraining the license to require free redistribution,
we eliminate the temptation to throw away many long-term gains in order to make
a few short-term sales dollars. If we didn't do this, there would be lots
of pressure for cooperators to defect.}\newline

Como se puede observar en este primer punto, y al contrario de lo que ocurría
con la \textit{Free Software Definition}, en esta definición se comienza a hilar
muy fino y tratando temas muy concretos. Este punto en concreto trata sobre
el hecho de que la licencia no debe restringir la distribución de un
\textit{software} como parte de una distribución de software agregado o global,
conteniendo programas de diferentes fuentes, al mismo tiempo que el creador
original tenga el derecho a exigir \textit{royalties}, como por ejemplo una
determinada cantidad económica por cada descarga, etc.\newline

{\bf 2. Source Code

The program must include source code, and must allow distribution in source code
as well as compiled form. Where some form of a product is not distributed with
source code, there must be a well-publicized means of obtaining the source code
for no more than a reasonable reproduction cost preferably, downloading via the
Internet without charge. The source code must be the preferred form in which a
programmer would modify the program. Deliberately obfuscated source code is not
allowed. Intermediate forms such as the output of a preprocessor or translator
are not allowed.}

\textit{Rationale: We require access to un-obfuscated source code because you
can't evolve programs without modifying them. Since our purpose is to make
evolution easy, we require that modification be made easy.}\newline

En este segundo punto se indica que será necesario incluir el código fuente en
la distribución del \textit{software}, y en caso de que no sea así, se debe
informar claramente de las diferentes maneras existentes para obtenerlo, nunca
por más de un precio razonable en concepto del coste de reproducción, o a través
de \textit{Internet} sin coste adicional. Otro aspecto a destacar en este punto
sobre referencias a aspectos concretos, es la referencia a que el código fuente
no se debe ofuscar de manera deliberada (cambiar los nombres de las variables
por otros nombres aleatorios, eliminar espacios y tabuladores alineando el
código fuente en una única línea de código, etc.) con tal de que la legibilidad
del código se mantenga para facilitar el estudio del mismo, etc.\newline

{\bf 3. Derived Works

The license must allow modifications and derived works, and must allow them to
be distributed under the same terms as the license of the original software.}

\textit{Rationale: The mere ability to read source isn't enough to support
independent peer review and rapid evolutionary selection. For rapid evolution to
happen, people need to be able to experiment with and redistribute
modifications.}\newline

La licencia debe permitir modificaciones y obras derivadas, y debe permitir su
redistribución bajo las mismas condiciones. Aquí se encuentra un matiz
importante, ya que recurriendo a aspectos de propiedad intelectual, se podría
``obligar a'', mientras que se prefiere utilizar ``te permito
hacerlo''. De hecho, cabe comentar la existencia de licencias de
\textit{Software Libre} que no obligan a mantener la misma licencia, de modo que
se pueden generar obras derivadas privativas, tal y como ocurre en el caso de
las licencias tipo \textit{Apache}.\newline

{\bf 4. Integrity of The Author's Source Code

The license may restrict source-code from being distributed in modified form
only if the license allows the distribution of "patch files" with the source
code for the purpose of modifying the program at build time. The license must
explicitly permit distribution of software built from modified source code. The
license may require derived works to carry a different name or version number
from the original software.}

\textit{Rationale: Encouraging lots of improvement is a good thing, but users
have a right to know who is responsible for the software they are using. Authors
and maintainers have reciprocal right to know what they're being asked to
support and protect their reputations.\newline Accordingly, an open-source
license must guarantee that source be readily available, but may require that it
be distributed as pristine base sources plus patches. In this way,
``unofficial'' changes can be made available but readily distinguished from the
base source.}\newline

Se permite la distribución y modificación del \textit{software}, pero de tal
modo que la obra original quede íntegra, mientras que las modificaciones se
realicen a modo de parches, pudiéndose realizar obras derivadas, pero no como la
obra original (caso de \textit{Apache}, y sus modificaciones como pueda ser el
ejemplo de \textit{Cherokee}). Este aspecto hace referencia a la práctica es
habitual consistente en el envío de parches al autor original para que los
incluya con tal de no tener que volver a aplicarlos con nada nueva
\textit{release} de dicho \textit{software}.\newline

{\bf 5. No Discrimination Against Persons or Groups

The license must not discriminate against any person or group of persons.}

\textit{Rationale: In order to get the maximum benefit from the process, the
maximum diversity of persons and groups should be equally eligible to contribute
to open sources. Therefore we forbid any open-source license from locking
anybody out of the process.\newline Some countries, including the United States,
have export restrictions for certain types of software. An OSD-conformant
license may warn licensees of applicable restrictions and remind them that they
are obliged to obey the law; however, it may not incorporate such restrictions
itself.}\newline

Este punto (así como el punto siguiente) hace referencia a la libertad 0
indicada por la \textit{Free Software Definition}, de tal modo que no se debe
discriminar a ninguna persona bajo ningún criterio (sexo, religión, raza,
nacionalidad, ideología política, etc.). Pero en cualquier caso, el hecho de que
el \textit{software} sea libre, no quiere decir que tus actos con el mismo sean
legales, como pueda ser la exportación de armas, etc.\newline

{\bf 6. No Discrimination Against Fields of Endeavor

The license must not restrict anyone from making use of the program in a
specific field of endeavor. For example, it may not restrict the program from
being used in a business, or from being used for genetic research.}

\textit{Rationale: The major intention of this clause is to prohibit license
traps that prevent open source from being used commercially. We want commercial
users to join our community, not feel excluded from it.}\newline

Haciendo referencia también a la libertad 0, la licencia no debe restringir el
uso del programa en un determinado campo de actividad.\newline

{\bf 7. Distribution of License

The rights attached to the program must apply to all to whom the program is
redistributed without the need for execution of an additional license by those
parties.}

\textit{Rationale: This clause is intended to forbid closing up software by
indirect means such as requiring a non-disclosure agreement.}\newline

En este punto, lo importante es que no se pueden incluir condiciones adicionales
que puedan restringir las libertades del \textit{software}.\newline

{\bf 8. License Must Not Be Specific to a Product

The rights attached to the program must not depend on the program's being part
of a particular software distribution. If the program is extracted from that
distribution and used or distributed within the terms of the program's license,
all parties to whom the program is redistributed should have the same rights as
those that are granted in conjunction with the original software distribution.}

\textit{Rationale: This clause forecloses yet another class of license
traps.}\newline

En este punto se indica que no se puede permitir la distribución como
\textit{software libre} para un determinado producto, pero no para otros (por
ejemplo, no se puede indicar que un \textit{software} se puede distribuir como
libre bajo una distribución \textit{Debian}, pero no bajo una distribución
\textit{Red Hat}). Es decir, no se pueden producir acuerdos específicos, sino
que las reglas deben ser las mismas para todos.\newline

{\bf 9. License Must Not Restrict Other Software

The license must not place restrictions on other software that is distributed
along with the licensed software. For example, the license must not insist that
all other programs distributed on the same medium must be open-source software.}

\textit{Rationale: Distributors of open-source software have the right to make
their own choices about their own software.\newline Yes, the GPL v2 and v3 are
conformant with this requirement. Software linked with GPLed libraries only
inherits the GPL if it forms a single work, not any software with which they are
merely distributed.}\newline

Para entender esta regla en su integridad, será necesario el estudio de la
licencia \textit{GPL} en el capítulo \ref{CHAP2:Licenses}.\newline

{\bf 10. License Must Be Technology-Neutral

No provision of the license may be predicated on any individual technology or
style of interface.}

\textit{Rationale: This provision is aimed specifically at licenses which
require an explicit gesture of assent in order to establish a contract between
licensor and licensee. Provisions mandating so-called "click-wrap" may conflict
with important methods of software distribution such as FTP download, CD-ROM
anthologies, and web mirroring; such provisions may also hinder code re-use.
Conformant licenses must allow for the possibility that (a) redistribution of
the software will take place over non-Web channels that do not support
click-wrapping of the download, and that (b) the covered code (or re-used
portions of covered code) may run in a non-GUI environment that cannot support
popup dialogues.}\newline

Se trata de otro punto peculiar dentro de la definición, según el cual, la
licencia no debe presuponer ninguna tecnología, cosa que se entiende
fácilmente en los dispositivos empotrados (lavadora, TDT, etc.), ya que
normalmente no se dispondrá de una interfaz adecuada (\textit{display}, botón de
\textit{OK} o aceptar, etc.) como para visualizar los términos de la licencia
del \textit{software} para aceptarlos o rechazarlos.

\subsection{Comparación de Diferentes Definiciones}

Llegados a este punto, debemos realizarnos la siguiente pregunta: ¿en qué se
diferencian las diversas definiciones de \textit{software libre}? Durante esta
sección, intentaremos responder a la pregunta.

Las diferencias no van más allá de un mero aspecto filosófico. Así, \textit{Free
Software} es el término empleado por aquellos que tratan sobre la parte más
política y cercana a los aspectos de libertad de expresión, mientras que el
término \textit{Open Source} se ha popularizado en los círculos de ámbito
comercial, quedando quizás más próximo a los aspectos puramente pragmáticos del
\textit{software libre}.

Sin embargo, \textit{Free Software} y \textit{Open Source} no son más que
maneras diferentes de referirse a una misma cosa, hasta el punto de que
prácticamente las podemos usar como sinónimos. De hecho, personas con diferentes
maneras de pensar, \textit{a priori}, respecto a un mismo tema, materializan sus
ideas en la práctica de una misma forma: las licencias de \textit{software
libre}. La mejor prueba para demostrar la equivalencia entre los términos es
que las listas de licencias aceptadas como de \textit{software libre} ofrecidas
por unas y otras instituciones son prácticamente exactas salvo en casos muy
puntuales. Precisamente esas excepciones, se deben a las diferentes maneras
posibles de interpretar una misma cosa, pero siendo siempre coincidentes en la
esencia.

En definitiva y a modo de conclusión, debemos tener en cuenta que \textit{Free
Software} y \textit{Open Source} no son dos cosas diferentes. De la misma manera
que tampoco lo son los términos \textit{Libre Software} (nacido igualmente para
evitar la confusión a la que puede llevar el término \textit{``free''} del
inglés, de modo que se adoptó la palabra del español) y el más reciente
\textit{FLOSS} (\textit{Free, Libre Open Source Software}) nacido como una
manera de unificar criterios con tal de dar origen a un término único como
forma de referirse y denotar al mundo del \textit{software libre}.
