\chapter{Modelos de Negocio}

En este capítulo la idea fundamental que se va a presentar es ¿Quién paga el
desarrollo de los programas?, ¿Por qué alguien paga por un software?, ¿Cómo
financiar proyectos y recursos?.


La financiación de los proyectos pueden llegar de diferentes maneras:

\begin{itemize}
  \item Financiación externa. Alguien financia el proyecto y decide cómo y en
  qué se gastan los recursos.
  \item Autofinanciados. La financiación proviene de las actividades de la
  empresa.
  \item Desarrollos con financiación indirecta. Participación de desarrolladores
  en proyectos pero que trabajan para otras empresas.
  \item Desarrollo para uso interno.
  \item Modelos mixtos. Una pequeña mezcla de los anteriores.
\end{itemize}

A continuación vamos a comentar más en detalle cada uno de los tipos.


\section{Financiación externa}

Como bien indica esta sección, la financiación proviene de una fuente externa
que por diversos motivos quiere que el software sea libre. En los siguientes
apartados vamos a poder ver las diferentes formas de financiación externa que
existen.

\subsection{Empresa pública}


La empresas públicas tiene diversos modelos de financiación, uno de ellos es
del tipo I+D y se decide que para poder financiar un proyecto, éste tendrá que
ser libre.

Otro modelo de financiación es crear proyectos competitivos, como por ejemplo
con los planes Avanza, que se evalúan según lo que pueden ofrecer a la
investigación. Éstos pueden ser software libre.

Muchas veces con este tipo de financiación no se busca recuperar la inversión,
aunque hay casos en los que se puede recuperar mediante subproductos, pero no
es la idea.

Las motivaciones para la financiación de este tipo de proyectos son variadas,
como científicas, promocionales, sociales o políticas.

Dentro de la financiación externa, existen algunas variantes de las cuales
hablaremos a continuación.

\subsection{Empresa privada sin ánimo de lucro}

Normalmente esta financiación está realizada por fundaciones u ONGs, ya bien
sea para producir software o para solucionar problemas mediante el desarrollo de
software.

La forma de financiar los proyectos es muy parecida a la que se usa en las
empresas públicas.


\subsection{Porque alguien necesita mejoras}
En este caso sí existe el ánimo de lucro. El proyecto en cuestión se financia
porque va a realizar mejoras a un software que necesito.
Esta financiación se puede hacer de varias maneras:
\begin{itemize}
  \item Pagando a una empresa externa.
  \item Pagando a una fundación.
  \item Contratando personal.
\end{itemize}

De cualquiera de las tres maneras se estaría financiando el proyecto del cual
finalmente saldremos beneficiados.


Uno de los casos es \emph{Corel}, empresa que quería portar todo su software a Linux para
poder competir con Microsoft. Y lo que decidieron es usar un emulador llamado
\emph{Wine}, y lo financiaron para que sus aplicaciones funcionaran bien en
Linux, y pagaron a una empresa llamada \emph{Macadamian} para que lo realizase.

\subsection{Indirecta}

La idea de la financiación indirecta es que financio un proyecto, no tanto
porque me interese el proyecto en sí, sino por los productos que puedan salir de
él. En el caso de Google, por ejemplo, que financia un proyecto de software
libre y obtiene un producto como Android.

Google se introduce en el mercado móvil para poder tener control y estar de
intermediarios de todo lo posible como publicidad, aplicaciones, etc\ldots
Crean un sistema operativo como Android, no para ganar dinero con Android, pero
si para no dejar que otros le quiten el poder de ser intermediarios y ser competitivos.
Buscan beneficios en productos relacionados con Android, no del propio Android.
Y el software libre es el instrumento para hacerlo. Intentar obtener céntimos
por cada compra/venta en el mundo con sus productos. La ganancia de Android, es
que lo pueda usar mucha gente y cuanta más gente mejor, bien en dinero o bien en
control.

En el caso de VISA/MASTERCARD hacen algo de este estilo, y la idea de Apple o
Google es ésta. RIM está creando WebOS con software libre pero es privativo.

Tenemos varios ejemplos:
\begin{itemize}
  \item Con Perl, crearon el software y se recuperó el dinero vendiendo libros.
  Cuantos más usuarios de Perl más gente compraría el libro. Y en vez de
  pagar por crear el libro, se pagó porque Perl fuese mejor y tuviese más
  versiones.
  \item Con el Hardware, VA (VA-Linux) vendía servidores, y lo que hace ahora es vender
servidores con Linux preinstalado y pagó para que las versiones de Linux
de sus servidores corriesen muy bien y sin problemas. Hay casos en los que se
pagan para que hagan el driver para ciertas tarjetas y cuando están disponibles
para Linux su venta aumenta.
\item Otro caso son las distribuciones de Linux, como Ubuntu, que se paga para
que la distribución sea conocida y mucha gente la utilice.
\end{itemize}



\section{Autofinanciados}

Con la autofinanciación soy yo el que invierte en el desarrollo y luego
trato de recuperarlo de alguna manera.

Existen varias formas de enfocar la autofinanciación y las veremos en este
apartado.

\subsection{Mejor conocimiento}

La idea es tratar de vender que tengo el mejor conocimiento de un producto, esto
puede hacer que después cuando haya que realizar un cambio pueda ofrecer precios
más baratos que si lo hace otro porque conozco muy bien el producto o bien para
intentar obtener una marca. Esto se puede dar o bien desarrollando el propio
programa o bien trabajando en otro software que ya existe.

Si tenemos un buen conocimiento de un producto podemos vender consultoría,
adaptación, integración, etc\ldots

Intentar que alguien use mucho el software y luego me paguen por hacer
modificaciones o añadir servicios.

Algunos casos, pueden ser:
\begin{itemize}
  \item Levanta, dan consultoría y soporte GNU/Linux y software libre en EEUU.
  \item Alcove, daba consultoría y consultoría estratégica para software libre
  en Europa.
\end{itemize}


\subsection{Mejor conocimiento con limitaciones}

El tener el mejor conocimiento tiene la limitación de que el resto también
puedan conseguirlo. Aunque existe otra forma que se llama tener
mejor conocimiento con limitaciones.

Este modelo intenta limitar el problema que teníamos anteriormente de que la
competencia también sea experta en el mismo producto que tú. Para ello se
utilizan licencias, en las que tienen una parte privativa y una parte libre. De
manera que la parte privativa esta más enfocada para dar servicio a empresas y
éstas tienen que pagar y una parte libre más enfocada para todo el mundo.
Normalmente la parte privada tendrá componentes o servicios que no tendrá la
libre, y así potencias que paguen la licencia.

El problema que tiene esto es que las comunidades, pueden crear las partes
que falten en la parte privativa y eso te hará perder marca y ser el productor
principal. Como es el caso de OpenOffice y LibreOffice.

Los proveedores de CRM hacen cosas de este estilo.


\subsection{Fuente de un programa}

Ser la fuente de un programa. Es la idea de tener una marca y ser el punto de
referencia sobre un producto. Es más terminos de imagen, como Sun con OpenOffice.

Otro ejemplo claro es IBM que financia Eclipse y esto da una sensación de que
IBM hace cosas serias.

Finalmente la financiación se termina rentabilizando en términos de imagen de
marca.

Algunos ejemplos más de esto son:
\begin{itemize}
 \item Abiword, intentó crear una suite ofimática, pero cuando OpenOffice se liberó,
pues no consiguieron seguir.
\item Evolution, RedCarpet, creada por Ximian, las regalaban en las
aplicaciones, finalmente se vendieron a Novel, para recuperar el dinero
invertido para crear Evolution. Mucha gente les conocía y estaban en una buena
posición dentro del mercado.
\item Zope, tenía un producto privativo, cuando fueron a pedir financiación, la
empresa les dijo que sólo les financiaba si lo hacían libre, de esta manera se
ha creado una gran comunidad alrededor de Zope.
\item Por último, el caso de Asterix, que se espera que sea el centro para la
gestión de centralitas para su uso en VoIP.
\end{itemize}



\subsection{Fuente de un programa con limitaciones}

Otra forma de poner limitaciones a la competencia es que primero puedo hacer el
producto privativo y cuando me interese lo libero, de esta manera la competencia
irá retrasada con respecto a mí.

Un caso claro es lo que está haciendo Google con la última versión de Android,
primero se la facilita a los fabricantes de móviles para que puedan adaptarlas a
su manera, y cuando éstos ya están disponibles en el mercado, libera la versión
para el resto del mundo.

Aunque hay que decir que esto no siempre funciona.

\subsection{Licencias especiales}
Es una variante del apartado anterior, lo único es que lo hago a la vez.
Pero la versión privativa me permite crear trabajos
derivados de ella y puedo ponerle la licencia que yo quiera. En cambio, la
versión libre siempre llevará una licencia GPL y si alguien realiza una versión
derivada, tendrá que añadir esta licencia.

Según el modelo de negocio dependerá mucho de las licencias que uses.

\subsection{Venta de marca}
Lo único que quiero es vender marca, del estilo de Red-Hat. Vende servicios
alrededor de su marca, sólo por tener la marca la gente ya va a pagar porque
tienes una imagen detrás que respalda el trabajo que realizas.

\section{Desarrollos sin financiación directa}
Es posible que existan proyectos en los que haya desarrolladores que sus
propias empresas les dejan trabajar para otros proyectos de software libre, bien
en su tiempo libre o bien ciertas horas al día.

Aunque no están financiados sí reciben alguna contribución, como parches por
alguna cierta funcionalidad, donación de máquinas como infraestructura,
donaciones propias, etc\ldots


\section{Desarrollos para uso internos}
Este tipo de financiación está basada en que yo necesito algo y lo desarrollo como
software libre, de esta manera al publicarse, el código estará más limpio y en un
futuro puedo encontrar colaboración.


